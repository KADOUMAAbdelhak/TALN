% !TEX TS-program = xelatex
% !TeX program = xelatex
% !TEX encoding = UTF-8
% !TEX spellcheck = fr

%=====================================================================
\ifx\wholebook\relax\else
	\documentclass{KodeBook}
	% !TEX TS-program = xelatex
% !TeX program = xelatex
% !TEX encoding = UTF-8
% !TEX spellcheck = fr

%\usepackage[T1]{fontenc}

%\usepackage[pdftex]{graphicx}

%\usepackage{listingsutf8}
%\usepackage{xcolor}
%\usepackage{times}
\usepackage{array}
\usepackage{natbib}
\usepackage{lscape}%to flip tables in a page
\usepackage{pdflscape}
\usepackage{longtable}
\usepackage{tabu}
\usepackage{wrapfig}
\usepackage{colortbl}
\usepackage{alltt}
\usepackage[french,lined]{algorithm2e}

\renewcommand{\cite}[1]{\citep{#1}}

%\usepackage[english]{babel}

\bibliographystyle{engdnat}%unsrtnat, plainnat

%\usepackage{pgf-umlcd}





\hypersetup{
	pdfkeywords={TALN; TAL; langue},
	pdfsubject={intelligence artificielle; traitement automatique de langages naturels}
}

\renewcommand{\UrlFont}{\ttfamily\footnotesize}

\DeclareAcronym{taln}{
	short = TALN ,
	long  = traitement automatique de langages naturels,
	class = abbrev
}

\DeclareAcronym{tal}{
	short = TAL ,
	long  = traitement automatique des langues,
	class = abbrev
}

\DeclareAcronym{ia}{
	short = IA ,
	long  = intelligence artificielle,
	class = abbrev
}

\DeclareAcronym{ibm}{
	short = IA ,
	long  = international business machines,
	class = abbrev
}

\DeclareAcronym{darpa}{
	short = DARPA ,
	long  = Defense Advanced Research Projects Agency,
	class = abbrev
}

\DeclareAcronym{ipa}{
	short = IPA ,
	long  = intelligent personal assistant,
	class = abbrev
}

\DeclareAcronym{iva}{
	short = IVA ,
	long  = intelligent virtual assistant,
	class = abbrev
}

\DeclareAcronym{ipa2}{
	short = IPA ,
	long  =  International Phonetic Alphabet,
	class = abbrev
}

%\makeglossaries

%\newacronym{oop}{OOP}{Object-oriented programming} 

	\begin{document}
		\mainmatter
	
\fi
%=====================================================================
\changegraphpath{../img/sem-lex/}
\chapter{Sémantique lexicale}

\begin{introduction}
	\lettrine{L}{e} 
\end{introduction} 

\begin{exampleblock}{Exemple de la polysémie en français}
	\begin{center}
		\Large\bfseries
		%		\begin{itemize}
		Je veux boire du \expword{café}.
		
		Je veux aller au \expword{café}.
		
		J'ai récolter du \expword{café}.
		%		\end{itemize}
	\end{center}
\end{exampleblock}

\begin{itemize}
	\item Est-ce que le mot ``\expword{café}" veut la même chose dans les trois phrases ?
	\item Comment peut-on savoir le sens d'un mot dans ce cas ?
	\item Les sens du mot ``\expword{café}" : \url{https://babelnet.org/search?word=caf\%C3\%A9&lang=FR}
\end{itemize}

%===================================================================================
\section{Bases de données lexicales}
%===================================================================================

\begin{itemize}
	\item Une base de données contenant le vocabulaire d'une langue
	\item Informations d'un mot : catégorie grammaticale, lemme, fréquence, etc.
\end{itemize}

\begin{figure}
	\centering 
	\hgraphpage[.5\textwidth]{exp-bd-lex_.pdf}
	\caption{Exemple des informations et leurs relations dans une base lexicale \cite{2019-white-al}}
\end{figure}

\subsection{Relations sémantiques}

\begin{itemize}
	\item \optword{Synonymie} : avoir des sens similaires dans un contexte donné
	\item \optword{Antonymie} : avoir des sens opposés dans un contexte donné
	\item Les relations taxonomiques (de classification)
	\begin{itemize}
		\item \optword{Hyponymie} : être plus spécifique qu'un autre sens. Il entraîne un relation \keyword{IS-A}. Ex. \expword{voiture IS-A véhicule} 
		\item \optword{Hyperonymie} : être plus générique qu'un autre sens. 
		\item \optword{Méronymie} : être une partie d'une chose. Ex. \expword{roue est un méronyme de voiture ; voiture est le holonyme de roue}
	\end{itemize}
\end{itemize}

\subsection{Wordnet}

\begin{figure}
	\hgraphpage{exp-wordnet_.pdf}
	\caption{Un exemple de WordNet généré par \url{http://wordnetweb.princeton.edu/perl/webwn}}
\end{figure}

\begin{itemize}
	\item Base de données lexicale pour l'anglais \cite{1995-miller}
	\item Trois parties : (1) noms (2) verbes (3) adjectifs et adverbes
	\item Un sens est représenté par un identifiant (\keyword{synset})
	\begin{itemize}
		\item \keyword{synset} (Synonyms set) : regroupe les mots du même sens 
		\item Ex. \expword{05659525 : reason\#3, understanding\#4, intellect\#2}
	\end{itemize}
	\item Un sens est défini par un glossaire (\keyword{gloss})
	\begin{itemize}
		\item Ex. \expword{05659525 : (the capacity for rational thought or inference or discrimination) "we are told that man is endowed with reason and capable of distinguishing good from evil"}
	\end{itemize}
	\item Un sens est marqué par une catégorie lexicographique (\keyword{supersense})
	\begin{itemize}
		\item Ex. \expword{05659525 : noun.cognition}
	\end{itemize}
	\item WordNet représente les relations sémantiques entre les sens
\end{itemize}

Catégories lexicographiques (supersense)
\begin{figure}
	\hgraphpage{wordnet-supersenses_.pdf}
	\caption{Les catégories lexicographiques des noms dans WordNet \cite{2019-jurafsky-martin}}
\end{figure}

Relations sémantiques
\begin{figure}
	\hgraphpage{wordnet-rel-nom_.pdf}\vspace{-9pt}
	\caption{Quelques relations des noms \cite{2019-jurafsky-martin}}
\end{figure}\vspace{-6pt}

\begin{figure}
	\hgraphpage{wordnet-rel-verbe_.pdf}\vspace{-9pt}
	\caption{Quelques relations des verbes \cite{2019-jurafsky-martin}}
\end{figure}

Quelques ressources
\begin{itemize}
	\item Quelques APIs
	\begin{itemize}
		\item NLTK (Python) : \url{https://www.nltk.org/howto/wordnet.html}
		\item JWI (Java) : \url{http://projects.csail.mit.edu/jwi}
		\item Wordnet (Ruby) : \url{https://github.com/wordnet/wordnet}
		\item OpenNlp (C\#) : \url{https://github.com/AlexPoint/OpenNlp}
	\end{itemize}
	\item Autres langues
	\begin{itemize}
		\item Global WordNet Association : \url{http://globalwordnet.org/resources/wordnets-in-the-world/}
		\item Open Multilingual Wordnet : \url{http://compling.hss.ntu.edu.sg/omw/}
	\end{itemize}
\end{itemize}

\subsection{Autres ressources}

\begin{itemize}
	\item \optword{FrameNet} 
	\begin{itemize}
		\item Basée sur la théorie ``cadre sémantique" (frame semantic)
		\item Un cadre peut être un évènement, une relation ou un entité avec ces participants 
		\item Ex. \expword{Le concept ``Cuisinier" implique une personne qui cuisine, la nourriture, un récipient et une source de chaleur}
		\item Chaque cadre est activé par un ensemble des \keyword{unités lexicales}. Ex. \expword{blanchir, bouillir, griller, dorer, mijoter, cuire}
	\end{itemize}
	\item \optword{VerbNet} 
	\begin{itemize}
		\item Une base lexicale pour verbes
		\item Elle inclue 30 rôles thématiques principaux 
		\item Les verbes sont organisés en classes
	\end{itemize}
	\item \textbf{On va revoir ces deux BD dans le chapitre suivant}
\end{itemize}

\begin{itemize}
	\item Réseaux sémantique multilingue
	\item \url{https://babelnet.org/}
\end{itemize}

\begin{figure}
	\hgraphpage{babelnet.png}
	\caption{Structure de BabelNet \cite{2012-navigli-ponzetto}}
\end{figure}

%===================================================================================
\section{Représentation vectorielle des mots}
%===================================================================================

\begin{itemize}
	\item Un mot peut être représenté en utilisant l'encodage \keyword{One-Hot} 
	\begin{itemize}
		\item Ne représente pas les relations sémantiques entre les mots : similarité (Ex. \expword{chat, chien}) et proximité (Ex. \expword{café, tasse})
		\item Comment représenter un document/phrase en utilisant cette représentation ?
	\end{itemize}
	
	\item \optword{Terme-document}
	\begin{itemize}
		\item On représente un terme par les documents le contiennent (ou l'inverse)
		\item Ex. \expword{Term frequency - innverse document frequency (Tf-Idf)}
	\end{itemize}
	
	\item \optword{Document-terme}
	\begin{itemize}
		\item On représente un terme par d'autres termes
	\end{itemize}
	
	\item \optword{Terme-concept-document}
	\begin{itemize}
		\item On représente les termes et les documents par un vecteur de concepts
		\item Ex. \expword{Analyse sémantique latente (LSA)}
		\item Bienvenu dans \keyword{le plongement lexical (word embedding)}
	\end{itemize}
	
\end{itemize}

\subsection{TF-IDF}

\begin{itemize}
	\item Le sens d'un document/phrase peut être représenté par ses mots 
	\item La fréquence d'un mot dans un document/phrase s'appelle \keywords{term frequency (TF)}
	\item Il y a des mots qui se répètent beaucoup, mais qui n'ont pas de sens ajouté (Ex. \expword{
		Prépositions})
	\item Les mots qui apparient dans plusieurs documents sont moins importants
\end{itemize}

\begin{block}{Calcul de TF-IDF}
	\[
	TF_d(t) =  |\{t_i \in d / t_i = t\}|
	\hskip2cm 
	DF_D(t) = |\{d \in D / t \in d\}|
	\]
	\[IDF_D(t) = \log_{10} \left( \frac{|\{d \in D\}|}{DF_D(t)} \right)\]
	\[TF\text{-}IDF_{d, D}(t) = TF_d(t) * IDF_D(t)\]
\end{block}

 Exemple d'une représentation avec TF
 
 \begin{exampleblock}{Exemple d'une collection de phrases}
 	\begin{itemize}
 		\item S1 : un ordinateur peut vous aider
 		\item S2 : il peut vous aider et il veut vous aider
 		\item S3 : il veut un ordinateur et un ordinateur pour vous
 	\end{itemize}
 \end{exampleblock}
 
 \begin{center}
 	\begin{tabular}{llllllllll}
 		\hline\hline
 		& un & ordinateur & peut & vous & aider & il & et & veut & pour \\
 		\hline
 		S1 & 1 & 1 & 1 & 1 & 1 & 0 & 0 & 0 & 0\\
 		S2 & 0 & 0 & 1 & 2 & 2 & 2 & 1 & 1 & 0\\
 		S3 & 2 & 2 & 0 & 1 & 0 & 1 & 1 & 1 & 1\\
 		\hline\hline
 	\end{tabular}
 \end{center}

Similarité cosinus
\begin{minipage}{.68\textwidth}
	\begin{itemize}
		\item $a$ et $b$ sont deux documents représentés par deux vecteurs $\overrightarrow{a}$ et $\overrightarrow{b}$ respectivement
		\item La représentation peut être les mots du vocabulaire (TF ou TF-IDF) ou d'autres caractéristiques du document
		\item La longueur des vecteurs est de $n$
	\end{itemize}
\end{minipage}
\begin{minipage}{.3\textwidth}
	\hgraphpage{exp-cos.pdf}
\end{minipage}

\begin{block}{Calcul de similarité cosinus entre deux vecteurs}
	\[
	Cos(\theta) = \frac{\overrightarrow{a} \overrightarrow{b}}{||\overrightarrow{a}||\, ||\overrightarrow{b}||}
	= \frac{\sum_{i=1}^{n} a_i b_i}{\sqrt{\sum_{i=1}^{n} a_i^2} \sqrt{\sum_{i=1}^{n} b_i^2}}
	\]
\end{block}

\subsection{Mot-Mot}

\begin{itemize}
	\item On peut représenter un mot par rapport aux autres mots du vocabulaire en utilisant la co-occurrence
	\item Pour représenter les mots d'un vocabulaire $ V $, on doit utiliser une matrice $|V| \times |V|$
	\item Chaque mot est représenté par $|V|$ mots appelés \keyword{le contexte}
	\item La co-occurrence peut être calculée par rapport aux documents, aux phrases ou des fenêtres auteur du mot
	\item La fenêtre peut être 4 mots avant et 4 mots après
\end{itemize}

Exemple avec un fenêtre de 1-1

\begin{exampleblock}{Exemple d'une collection de phrases}
	\begin{itemize}
		\item S1 : un ordinateur peut vous aider
		\item S2 : il peut vous aider et il veut vous aider
		\item S3 : il veut un ordinateur et un ordinateur pour vous
	\end{itemize}
\end{exampleblock}

\begin{center}
	\scriptsize
	\begin{tabular}{llllllllll}
		\hline\hline
		& un & ordinateur & peut & vous & aider & il & et & veut & pour \\
		\hline
		un & 0 & 3 & 0 & 0 & 0 & 0 & 1 & 1 & 0\\
		ordinateur & 3 & 0 & 1 & 0 & 0 & 0 & 1 & 0 & 1\\
		peut & 0 & 1 & 0 & 2 & 0 & 1 & 0 & 0 & 0\\
		vous & 0 & 0 & 2 & 0 & 3 & 0 & 0 & 1 & 1\\
		aider & 0 & 0 & 0 & 3 & 0 & 0 & 1 & 0 & 0\\
		il & 0 & 0 & 1 & 0 & 0 & 0 & 1 & 2 & 0\\
		et & 0 & 0 & 0 & 0 & 1 & 1 & 0 & 0 & 0\\
		veut & 1 & 0 & 0 & 1 & 0 & 2 & 0 & 0 & 0\\
		pour & 0 & 1 & 0 & 1 & 0 & 0 & 0 & 0 & 0\\
		\hline\hline
	\end{tabular}
\end{center}

Similarité cosinus

\begin{itemize}
	\item On peut calculer la similarité entre deux mots
	\item Une mesure de similarité est cosinus (vue précédemment)
\end{itemize}

\hgraphpage{exp-word-v1_.pdf}

\begin{figure}
	\caption{Un exemple des vecteurs de co-occurrence à partir de Wikipedia et une visualisation de deux mots \cite{2019-jurafsky-martin} }
\end{figure}
\begin{minipage}{.4\textwidth}
	\hgraphpage{exp-word-v2_.pdf}
\end{minipage}

\subsection{Analyse sémantique latente (LSA)}

\begin{itemize}
	\item La matrice terme-document $X[N, M]$ 
	\begin{itemize}
		\item dimension trop grande
		\item beaucoup de zéros (voir l'exemple de Tf-Idf)
	\end{itemize}
	\item On peut utiliser $L$ concepts pour représenter
	\begin{itemize}
		\item les termes $T[N, L]$
		\item les documents $D[M, L]$
	\end{itemize}
	\item $L \le \min(N, M)$
	\item $X$ peut être décomposée en utilisant \keyword{décomposition en valeurs singulières (SVD)}
	\begin{itemize}
		\item $X = T \times S \times D^\top$
	\end{itemize}
\end{itemize}

Formulation SVD
\begin{itemize}
	\item $T^\top T = \mathbb{I}_{N \times N}$ 
	\item $D^\top D = \mathbb{I}_{M \times M}$ 
	\item SVD peut être résolue en utilisant par exemple l'\keyword{algorithme de Lanczos}, la \keyword{décomposition QR}
\end{itemize}

\begin{block}{Décomposition en valeurs singulières (SVD) de la matrice terme-document}
	\scriptsize\bfseries
	\[
	\overbrace{
		\begin{bmatrix}
		x_{11} & \ldots & \ldots & \ldots & x_{1M} \\ 
		\vdots & \ddots & \ddots & \ddots &\vdots \\
		\vdots & \ddots & \ddots & \ddots &\vdots \\
		x_{N1} & \ldots & \ldots & \ldots & x_{NM} \\ 
		\end{bmatrix}
	}^{X \text{ : terme-document}}
	=
	\overbrace{
		\left[
		\begin{bmatrix}
		t_{11} \\ 
		\vdots \\
		\vdots \\
		t_{N1} \\ 
		\end{bmatrix}
		\begin{matrix}
		\ldots \\ 
		\end{matrix}
		\begin{bmatrix}
		t_{L1} \\ 
		\vdots \\
		\vdots \\
		t_{L1} \\ 
		\end{bmatrix}
		\right]
	}^{T \text{ : terme-concept}}
	\times 
	\overbrace{
		\begin{bmatrix}
		s_{11} & \ldots & 0 \\
		0 & \ddots & 0 \\
		0 & \ldots & s_{LL} \\
		\end{bmatrix}
	}^{S \text{ : concept-concept}}
	\times 
	\overbrace{
		\begin{bmatrix}
		\begin{bmatrix}
		d_{11} & \ldots & \ldots & \ldots & d_{1M} \\
		\end{bmatrix}\\
		\vdots \\
		\begin{bmatrix}
		d_{11} & \ldots & \ldots & \ldots & d_{1M} \\
		\end{bmatrix}\\
		\end{bmatrix}
	}^{D^\top \text{ : concept-document}}
	\]
	
\end{block}

%===================================================================================
\section{Word embedding}
%===================================================================================

\begin{itemize}
	\item Les représentations document-mot et mot-mot basées sur la co-occurrence occupent un grand espace mémoire 
	\item Elles sont difficiles à gérer 
	\item Dans \keyword{LSA}, nous avons vu que la représentation d'un mot est compactée vers un petit vecteur de nombres réels
	\item Ceci est appelé : \keyword{Word embedding} ou, en français, \keyword{Plongement lexical}
	\item Dans ce cours, on se focalise sur le Word embedding basé sur les réseaux de neurones
	\item Cette technique permet d'apprendre à représenter des relations entre mots
\end{itemize}

\subsection{Word2vec}

\begin{itemize}
	\item Dans le modèle Mot-Mot, nous avons vu la notion du contexte 
	\begin{itemize}
		\item Un mot est représenté par les mots qui les entourent
		\item Un vecteur de fréquence de co-occurrence
		\item La longueur du vecteur est la taille du vocabulaire (trop large)
		\item La co-occurrence d'un mot avec un petit sous-ensemble de mots (beaucoup de zéros)
	\end{itemize}
	\item Word2vec est un outil fourni par Google implémentant deux méthodes de \keyword{Word embedding} \cite{2013-mikolov-al}
	\begin{itemize}
		\item \optword{CBOW} : Continuous Bag-of-Words
		\item \optword{Skip-gram} : Continuous Skip-gram
	\end{itemize}
	\item Encoder les mots sur un petit vecteur (50-1000) en se basant sur le contexte (en utilisant un encodeur-décodeur)
\end{itemize}

Word2vec : CBOW
\begin{minipage}{.58\textwidth}
	\begin{itemize}
		\item Le contexte ici est deux mots avant et deux après
		\item Inférer le mot $w_i$ étant donné son contexte
	\end{itemize}
	\begin{block}{Fonction du coût à minimiser}
		\[%
		J(\theta) = \frac{-1}{V} \sum_{i=1}^{V} p(w_i |w_{i-2}, w_{i-1}, w_{i+1}, w_{i+2})
		\]
	\end{block}
\end{minipage}
\begin{minipage}{.08\textwidth}
\end{minipage}	
\begin{minipage}{.4\textwidth}
	\hgraphpage{word2vec-cbow.pdf}
\end{minipage}

Word2vec : Skip-gram
\begin{minipage}{.58\textwidth}
	\begin{itemize}
		\item Le contexte ici est deux mots avant et deux après
		\item Inférer le contexte étant donné le mot $w_i$
	\end{itemize}
	\begin{block}{Fonction du coût à minimiser}
		\[%
		J(\theta) = \frac{-1}{V} \sum_{i=1}^{V} \sum_{j= i-2; j \ne i}^{i+2} p(w_j |w_i)
		\]
	\end{block}
\end{minipage}
\begin{minipage}{.08\textwidth}
\end{minipage}
\begin{minipage}{.4\textwidth}
	\hgraphpage{word2vec-skip.pdf}
\end{minipage}

\subsection{GloVe}

\begin{itemize}
	\item \keyword{GloVe} : Global Vectors
	\item une méthode développée par Stanford \cite{2014-pennington-al}
	\item essaye d'exploiter les deux approches : matrice mot-mot (comme LSA) et apprentissage par contexte (comme CBOW)
	\item $X[V, V]$ est la matrice Mot-Mot où $X_{ij}$ est le nombre des occurrence du mot $w_j$ dans le contexte de $w_i$, et $X_i$ est le nombre d'occurrences de $w_i$ dans le corpus
	\item $P_ij= \frac{X_{ij}}{X_i}$ est la probabilité d'occurrence de $w_j$ dans le contexte de $w_i$
\end{itemize}

Motivation
\begin{minipage}{.5\textwidth}
	\begin{itemize}
		\item On veut trouver la relation entre $w_i = ice$ et $w_j = steam$
		\item Ceci peut être représenté par un vecteur des mots $w_j$ 
		\item On calcule le ratio $\frac{P_ik}{Pjk}$
	\end{itemize}
\end{minipage}
\begin{figure}
	\hgraphpage{exp-glove_.pdf}
	\caption{Exemple de probabilités et un ratio \cite{2014-pennington-al}}
\end{figure}

\begin{itemize}
	\item la valeur du ratio doit suivre les relations $R(w_k, w_i)$ et $R(w_k, w_j)$
	\begin{itemize}
		\item Si $R(w_k, w_i) \wedge \neg R(w_k, w_j)$, le ratio sera grand. Ex. \expword{solid}
		\item Si $\neg R(w_k, w_i) \wedge R(w_k, w_j)$, le ratio sera petit. Ex. \expword{gas}
		\item Si $R(w_k, w_i) \wedge R(w_k, w_j)$, le ratio tend vers $1$ . Ex. \expword{water}
		\item Si $\neg R(w_k, w_i) \wedge \neg R(w_k, w_j)$, le ratio tend vers $1$ . Ex. \expword{fashion}
	\end{itemize}
	\item On veut entraîner une fonction $F$ sur les représentations de $w_i$, $w_j$ et $\tilde{w_k}$
	\vspace{-6pt}\[F(w_i, w_j, \tilde{w_k}) = \frac{P_ik}{P_jk}\]
\end{itemize}

Formulation (raisonnement)
\begin{itemize}
	\item On peut restreindre la fonction par utilisation d'une soustraction
	\vspace{-6pt}\[F(w_i - w_j, \tilde{w_k}) = \frac{P_ik}{P_jk}\]
	
	\item On peut transformer l'argument à un scalaire par multiplication matricielle
	\vspace{-6pt}\[F((w_i - w_j)^\top \tilde{w_k}) = \frac{P_ik}{P_jk}\]
	
	\item On veut : $w \leftrightarrow \tilde{w}$ et $X \leftrightarrow X^\top$. Donc, $F$ doit être un homomorphisme entre $(\mathbb{R}, +)$ et $(\mathbb{R}_{>0}, \times)$
	\vspace{-6pt}\[F((w_i - w_j)^\top \tilde{w_k}) = \frac{F(w_i^\top \tilde{w_k})}{F(w_j^\top \tilde{w_k})}\]
	
	\item Donc, $F(w_i^\top \tilde{w_k}) = P_ik = \frac{X_{ik}}{X_i}$
	
	\item La solution est $F = exp$. Donc $w_i^\top \tilde{w_k} = \log X_{ik} - \log X_i$
\end{itemize}

Formulation (version finale)
\begin{itemize}
	\item la valeur $log X_i$ est indépendante de $k$. Donc, on peut entraîner un biais $b_i$ sur $w_i$. Pour respecter la symétrie, il faut entraîner un biais $\tilde{b_k}$ sur $\tilde{w_k}$ 
	\item La fonction objective $J$ utilisée est \keyword{la méthode des moindres carrés}
	\begin{itemize}
		\item le problème est que $J$ ne doit pas pondérer les co-occurrences de la même façon : les co-occurrences rares doivent avoir moins d'impact sur $J$
		\item un solution est de pondérer $J$ avec une fonction sur $X_{ij}$
	\end{itemize}
\end{itemize}

\vspace{-6pt}
\begin{block}{Formulation de la méthode GloVe}\vspace{-6pt}
	\[w_i^\top \tilde{w_j} + b_i + \tilde{b_j} = \log X_{ij} \]
	\[J(\theta) = \sum_{i=1}^{V} \sum_{j=1}^{V} f(X_{ij}) (w_i^\top \tilde{w_j} + b_i + \tilde{b_j} - \log X_{ij})^2\]
	\vspace{-12pt}\[f(x) = \begin{cases}
	\frac{x}{x_{max}} & \text{si } x < x_{max} \\
	1 & \text{ sinon}
	\end{cases}\]
	\vspace{-6pt}\end{block}

Architecture
\begin{center}
	\hgraphpage{glove.pdf}
\end{center}

\subsection{Embeddings contextuels}

\begin{exampleblock}{Exemple de polysémie (mot avec plusieurs sens)}
	\begin{itemize}
		\item Le meilleur préservatif contre les \expword{souris} est un chat. (animal)
		\item La \expword{souris} pour ordinateur est un système de pointage. (dispositif)
		\item J'adore la \expword{souris}, c'est mon morceau favori. (Partie du gigot de mouton)
	\end{itemize}
\end{exampleblock}

\begin{itemize}
	\item Les représentations de word embedding comme Word2vec et GloVe, peuvent-elles refléter des relations entre le mot \expword{souris} et les mots chat, ordinateur et morceau ?
	\item Peuvent-elles différencier entre les différents sens du mot \expword{souris} ?
\end{itemize}

ELMo

\begin{minipage}{.65\textwidth}
	\begin{itemize}
		\item \keyword{ELMo} : Embeddings from Language Models
		\item \url{http://allennlp.org/elmo}
		\item \cite{2018-peters-al}
		\item Caractéristiques de la représentation
		\begin{itemize}
			\item Contextuelle :  prend en considération tout le contexte (phrase)
			\item Profonde : combine tous les résultats des couches du réseau de neurones
			\item Basée sur les caractères : prise en considération des caractéristiques morphologiques et des mots hors vocabulaire
		\end{itemize}
	\end{itemize}
\end{minipage}
\begin{minipage}{.33\textwidth}
	\vspace{2cm}
	\hgraphpage{Elmo-img.jpg}
\end{minipage}

\hgraphpage{elmo-arch.pdf}

\begin{itemize}
	\item Étant donné un mot $w_k$, on calcule sa représentation $x_k$ en se basant sur les caractères qui lui composent (voir \cite{2015-kim-al})
	\item On utilise $L$ couches des cellules LSTM bidirectionnelles 
	\item Pour chaque  phrase de $N$ mot, on essaye de maximiser\\ 
	\begin{center}
		$\sum_{k=1}^{N} 
		\log P(w_k | w_1,\ldots,w_{k-1}; \Theta_x, \overrightarrow{\Theta}_{LSTM}, \Theta_s)
		+
		\log P(w_k | w_{k+1},\ldots,w_{N}; \Theta_x, \overleftarrow{\Theta}_{LSTM}, \Theta_s)
		$
	\end{center}
	
	\item $R_k = \{x_k^{LM}, \overrightarrow{h}_{LM}^{k, j}, \overleftarrow{h}_{LM}^{k, j} | j= 1 \ldots L \}
	= \{h_{LM}^{k, j} | j= 0 \ldots L \}
	$
	
	\item Pour intégrer ELMo avec une tâche  $task$, on entraîne des paramètres $\Theta^{task}$ afin d'inférer une représentation liée à la tâche \\
	\begin{center}
		$ELMo_k^{task} = E(R_k; \Theta^{task}) = \gamma^{task} \sum_{j=0}^{L} s_j^{task} h_{LM}^{k, j}$
	\end{center}
\end{itemize}

BERT
\begin{minipage}{.68\textwidth}
	\begin{itemize}
		\item \keyword{BERT} : Bidirectional Encoder Representations from Transformers
		\item \url{https://github.com/google-research/bert}
		\item \cite{2019-devlin-al}
		\item Caractéristiques de la représentation
		\begin{itemize}
			\item Contextuelle :  prend en considération tout le contexte (phrase)
			\item Totalement bidirectionnelle : prend en considération les mots avant et après 
			\item Basée sur les tokens : les mots sont séparés en radical + infixes
		\end{itemize}
	\end{itemize}
\end{minipage}
\begin{minipage}{.3\textwidth}
	\vspace{.5cm}
	\hgraphpage{Bert-img.png}
\end{minipage}

\hgraphpage{bert-arch.pdf}

\begin{itemize}
	\item Le texte est séparé en tokens en utilisant \keyword{Wordpiece} (voir \cite{2016-wu-al})
	\item En entrée une séquence de tokens de longueur maximale $T = 512$
	\begin{itemize}
		\item Le premier token est un marqueur spécial ``\keyword{[CLS]}" utilisé pour la classification
		\item La séquence représente deux phrases séparées par un token ``\keyword{[SEP]}"
	\end{itemize}
	\item Chaque token est représenté par 3 embeddings de taille $N$ 
	\begin{itemize}
		\item \optword{Embedding de token} : transformation d'une vocabulaire de taille $V$ vers un vecteur de taille $N$
		\item \optword{Embedding de position} : transformation de la position du token dans la phrase sur une taille max $T$ vers un vecteur de taille $N$
		\item \optword{Embedding de segment} : transformation du segment du token (phrase1 ou phrase2) sur une taille de $2$ vers un vecteur de taille $N$
	\end{itemize}
\end{itemize}

\begin{itemize}
	\item Le modèle est basé sur \keyword{Transformer} (voir \cite{2017-vaswani-al})
	\item Il est entraîné sur deux tâches
	\item \optword{Modèle de langage masqué}
	\begin{itemize}
		\item masquer aléatoirement 15\% des tokens d'une phrase et essayer de les inférer
		\item on utilise un token spécial \keyword{[MASK]}
		\item puisque ce token n'apparait pas dans l'étape de réglage, on l'utilise pour 80\% des remplacements
		\item 10\% avec un token quelconque et 10\% sans changement 
	\end{itemize}
	\item \optword{Prédiction de la phrase suivante}
	\begin{itemize}
		\item Prédire si la deuxième phrase suit la première 
		\item Le résultat est dans la sortie du token ``\keyword{[CLS]}" ($CLS \in \{IsNext, NotNext\}$)
	\end{itemize}
	\item Le modèle pré-entraîné peut être entraîné sur une tâche définie
\end{itemize}

\begin{figure}
	\hgraphpage{bert-taches1_.pdf}
	\caption{Illustration de réglage de BERT sur des tâches différentes \cite{2019-devlin-al}
		(a) Relation entre deux phrases comme la similarité (b) Classification d'une phrase comme analyse de sentiments
	}
\end{figure}

\begin{figure}
	\hgraphpage{bert-taches2_.pdf}
	\caption{Illustration de réglage de BERT sur des tâches différentes \cite{2019-devlin-al}
		(a) Tâches de Question-réponse (b) Les tâches de tagging comme reconnaissance des entités nommées
	}
\end{figure}

\subsection{Évaluation des modèles}

Évaluation intrinsèque
\begin{itemize}
	\item  WordSimilarity-353 Test Collection \cite{2002-finkelstein-al}
	\begin{itemize}
		\item Des similarités entre mots annotés manuellement (0 et 10)
		\item On calcule la corrélation de Spearman entre les similarités basées sur les représentations (similarité cosinus) et les similarités manuelles
	\end{itemize}
	
	\item SimLex-999 \cite{2015-hill-al}
	\begin{itemize}
		\item Tester la similarité (\expword{coast, shore}) et pas l'association (\expword{clothes, closet})
	\end{itemize}
	
	\item Analogies de mots \cite{2013-mikolov-al2}
	\begin{itemize}
		\item Dataset de la forme $(w_{i1}:w_{j1} :: w_{i2}:w_{j2})$
		\item Tester la capacité des embeddings à représenter des relations $w_{j2} = w_{i1} - w_{i2} + w_{j1}$
		\item Ex. \expword{(King:Queen :: Man:Woman) \textrightarrow King - Man + Woman = Queen}
	\end{itemize}
	
\end{itemize}

Évaluation extrinsèque et biais

\begin{itemize}
	\item  \optword{Évaluation extrinsèque} : Évaluer les modèles en se basant sur une tâche 
	\begin{itemize}
		\item Évaluer un nouveau modèle par rapport à un autre connu en fonction d'une tâche
		\item Ex. GloVe surpasse LSA dans la tâche de reconnaissance des entités nommées \cite{2014-pennington-al}
	\end{itemize}
	\item \optword{Biais} : Les modèles peuvent apprendre des analogies biaisées
	\begin{itemize}
		\item Apprendre des stéréotypes comme \expword{she} avec \expword{homemaker, nurse, receptionist} et \expword{he} avec \expword{maestro, skipper, protege} \cite{2017-caliskan-al}
		\item Ceci peut affecter la performance de certaines tâches. 
		\item Ex. \expword{La résolution des anaphores peut échouer à lier le pronoms ``she" avec ``doctor"}
	\end{itemize}
\end{itemize}

%===================================================================================
\section{Désambigüisation lexicale}
%===================================================================================

\begin{itemize}
	\item Désambigüisation lexicale : Word Sense Disambiguation (WSD)
	\item Sélectionner le sens correct d'un mot 
	\item Utile pour plusieurs tâches 
	\begin{itemize}
		\item Analyse syntaxique. 
		
		\expword{I \underline{fish} in the river} (Verbe ou nom ?)
		
		\expword{The \underline{fish} was too big} (Verbe ou nom ?)
		
		\item Traduction automatique. 
		
		\expword{I withdrawed money from the \underline{bank}} (``banque" ou ``rive" ?)
		
		\expword{I fish on the \underline{bank}} (``banque" ou ``rive" ?)
	\end{itemize}
\end{itemize}

\subsection{Basée sur des bases de connaissance}

\begin{block}{Algorithme de Lesk}
	\footnotesize\vspace{-3pt}
	\begin{algorithm}[H]
		\Donnees{un mot $w$; une phrase $s$ contenant $w$}
		\Res{Le sens de $w$}
		
		meilleur\_sens \textleftarrow plus fréquent parmi les sens de $w$\;
		superposition\_max \textleftarrow 0\;
		contexte \textleftarrow l'ensemble des mots de $s$\; 
		
		\PourTous{sens $w_i$ de $w$}{ 
			signature \textleftarrow l'ensemble des mots dans le \textbf{gloss} et exemples du sens $w_i$\;
			superposition \textleftarrow nombre des mots en commun entre \textbf{contexte} et \textbf{signature}\;
			\Si{superposition $>$ superposition\_max}{
				superposition\_max \textleftarrow superposition\;
				meilleur\_sens \textleftarrow $w_i$\;
			}
		}
		
		\Retour meilleur\_sens \;
		\vspace{-3pt}
	\end{algorithm}
\end{block}

 A base de graphes
 
\begin{itemize}
	\item Exemple : \url{http://babelfy.org/} \cite{2014-moro-al}
	\item \optword{Étape 1 : Construction des signatures sémantiques}
	\begin{itemize}
		\item Attribuer un poids à chaque arc reliant deux concepts en se basant sur le nombre des triangles qui les relient (à partir d'un réseau sémantique)
		\item Calculer la probabilité d'un concept sachant un autre en se basant sur ces poids
		\item Minimiser le graphe en utilisant ``\keyword{Random walk with restart}"
	\end{itemize}
	\item \optword{Étape 2 : Identification des candidats}
	\begin{itemize}
		\item Appliquer l'étiquetage morphosyntaxique sur le texte d'entrée
		\item Extraire tous les sens des mots ou des expressions
	\end{itemize}
	\item \optword{Étape 3 : Désambigüisation des candidats}
	\begin{itemize}
		\item Construire un graphe en se basant sur la signature sémantique et les candidats
		\item Créer un sous-graphe en éliminant les liens faibles
	\end{itemize}
\end{itemize}

\subsection{Basée sur l'apprentissage automatique}

Apprentissage automatique supervisé

\begin{itemize}
	\item Utiliser un corpus annoté (Ex. \expword{SemCor})
	\item Chaque mot est suivi par le numéro de son sens dans une base lexicale (WordNet)
	\item Ex. \expword{You will find9 that avocado1 is1 unlike1 other1 fruit1 you have ever1 tasted2}
	\item En sortie, on aura tous les sens des mots du vocabulaire encodés avec One-Hot
	\item Les caractéristiques peuvent être les mots qui entourent le mot en question
	\item La désambigüisation de tous les mots d'un texte est similaire à la tâche d'étiquetage morphosyntaxique
\end{itemize}

Embeddings contextuels

\begin{minipage}{.68\textwidth}
	\begin{itemize}
		\item Algorithme du plus proche voisin (1-nearest-neighbor algorithm) 
		\item Utiliser un embedding contextuel (ELMo, BERT, etc.)
		\item Entraîner le modèle sur un corpus annoté pour avoir le embedding de chaque token
		\item Pour avoir le embedding d'un sens $v_s$, on fait la moyenne des embeddings $c_i$ qui l'appartiennent
		\[ v_s = \frac{1}{n} \sum_{i=1}^{n} c_i \] 
	\end{itemize}
\end{minipage}
\begin{figure}
	\hgraphpage{exp-wsd-nn_.pdf}
	\caption{Exemple de désambigüisation avec le plus proche voisin \cite{2019-jurafsky-martin}}
\end{figure}


\begin{discussion}



\end{discussion}

%=====================================================================
\ifx\wholebook\relax\else
% \cleardoublepage
% \bibliographystyle{../use/ESIbib}
% \bibliography{../bib/RATstat}
	\end{document}
\fi
%=====================================================================
