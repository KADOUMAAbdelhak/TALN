% !TEX TS-program = xelatex
% !TeX program = xelatex
% !TEX encoding = UTF-8
% !TEX spellcheck = fr

%\usepackage[T1]{fontenc}

%\usepackage[pdftex]{graphicx}

%\usepackage{listingsutf8}
%\usepackage{xcolor}
%\usepackage{times}
\usepackage{array}
\usepackage{natbib}
\usepackage{lscape}%to flip tables in a page
\usepackage{pdflscape}
\usepackage{longtable}
\usepackage{tabu}
\usepackage{wrapfig}
\usepackage{colortbl}
\usepackage{alltt}
\usepackage[french,lined]{algorithm2e}

\renewcommand{\cite}[1]{\citep{#1}}

%\usepackage[english]{babel}

\bibliographystyle{engdnat}%unsrtnat, plainnat

%\usepackage{pgf-umlcd}





\hypersetup{
	pdfkeywords={TALN; TAL; langue},
	pdfsubject={intelligence artificielle; traitement automatique de langages naturels}
}

\renewcommand{\UrlFont}{\ttfamily\footnotesize}

\DeclareAcronym{taln}{
	short = TALN ,
	long  = traitement automatique de langages naturels,
	class = abbrev
}

\DeclareAcronym{tal}{
	short = TAL ,
	long  = traitement automatique des langues,
	class = abbrev
}

\DeclareAcronym{ia}{
	short = IA ,
	long  = intelligence artificielle,
	class = abbrev
}

\DeclareAcronym{ibm}{
	short = IA ,
	long  = international business machines,
	class = abbrev
}

\DeclareAcronym{darpa}{
	short = DARPA ,
	long  = Defense Advanced Research Projects Agency,
	class = abbrev
}

\DeclareAcronym{ipa}{
	short = IPA ,
	long  = intelligent personal assistant,
	class = abbrev
}

\DeclareAcronym{iva}{
	short = IVA ,
	long  = intelligent virtual assistant,
	class = abbrev
}

\DeclareAcronym{ipa2}{
	short = IPA ,
	long  =  International Phonetic Alphabet,
	class = abbrev
}

%\makeglossaries

%\newacronym{oop}{OOP}{Object-oriented programming} 
