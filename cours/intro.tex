% !TEX TS-program = xelatex
% !TeX program = xelatex
% !TEX encoding = UTF-8
% !TEX spellcheck = fr

%=====================================================================
\ifx\wholebook\relax\else
	\documentclass{KodeBook}
	% !TEX TS-program = xelatex
% !TeX program = xelatex
% !TEX encoding = UTF-8
% !TEX spellcheck = fr

%\usepackage[T1]{fontenc}

%\usepackage[pdftex]{graphicx}

%\usepackage{listingsutf8}
%\usepackage{xcolor}
%\usepackage{times}
\usepackage{array}
\usepackage{natbib}
\usepackage{lscape}%to flip tables in a page
\usepackage{pdflscape}
\usepackage{longtable}
\usepackage{tabu}
\usepackage{wrapfig}
\usepackage{colortbl}
\usepackage{alltt}
\usepackage[french,lined]{algorithm2e}

\renewcommand{\cite}[1]{\citep{#1}}

%\usepackage[english]{babel}

\bibliographystyle{engdnat}%unsrtnat, plainnat

%\usepackage{pgf-umlcd}





\hypersetup{
	pdfkeywords={TALN; TAL; langue},
	pdfsubject={intelligence artificielle; traitement automatique de langages naturels}
}

\renewcommand{\UrlFont}{\ttfamily\footnotesize}

\DeclareAcronym{taln}{
	short = TALN ,
	long  = traitement automatique de langages naturels,
	class = abbrev
}

\DeclareAcronym{tal}{
	short = TAL ,
	long  = traitement automatique des langues,
	class = abbrev
}

\DeclareAcronym{ia}{
	short = IA ,
	long  = intelligence artificielle,
	class = abbrev
}

\DeclareAcronym{ibm}{
	short = IA ,
	long  = international business machines,
	class = abbrev
}

\DeclareAcronym{darpa}{
	short = DARPA ,
	long  = Defense Advanced Research Projects Agency,
	class = abbrev
}

\DeclareAcronym{ipa}{
	short = IPA ,
	long  = intelligent personal assistant,
	class = abbrev
}

\DeclareAcronym{iva}{
	short = IVA ,
	long  = intelligent virtual assistant,
	class = abbrev
}

\DeclareAcronym{ipa2}{
	short = IPA ,
	long  =  International Phonetic Alphabet,
	class = abbrev
}

%\makeglossaries

%\newacronym{oop}{OOP}{Object-oriented programming} 

	\begin{document}
		\mainmatter
	
\fi
%=====================================================================

\chapter*{Introduction}
\addcontentsline{toc}{chapter}{Introduction}




\section*{Cadre général}

Ce support est destiné aux étudiants du deuxième année cycle supérieure (2CS) ainsi que les étudiants du Master.
A l'école nationale Supérieure d'Informatique (ESI) d'Alger, la formation passe par un cycle préparatoire qui dure
2 années suivies par un concours afin de passer au cycle supérieur.
Ce dernier prend 3 années dont la première est un socle commun et les deux dernières représentent une formation de spécialité.
La dernière année est réservée au projet de fin d'étude où l'étudiant peuvent suivre une formation accélérer pour obtenir un diplôme du master en parallèle avec celui d'ingénieur.
Donc, la formation la plus proche au 2CS de l'ESI est Master 2 des universités.

Le traitement automatique du langage naturel (TALN) est souvent enseigné comme un module des spécialités ``intelligence artificielle" et ``science des données".
Il fait partie des branches de l'intelligence artificielle comme la vision par ordinateur, la robotique, le raisonnement par machine, etc.
Cela ne veut pas dire que les autres spécialités n'auront pas besoin de ce module ; il est seulement moins important que d'autres modules dans ces mêmes spécialités.
En génie logiciel, on peut par exemple exploiter ce domaine afin de concevoir un système qui construit toute une architecture logiciel à partir d'une description textuelle.
Le langage naturel fait partie des modes d'interaction homme-machine.
En système d'information, les techniques de ce domaine sont souvent utilisées pour la recherche et l'extraction d'information.
L'extraction de données est une phase d'un système ETL (Extract, Transform, Load) où les données peuvent être textuelles (non structurées).
En systèmes et réseaux, les différentes attaques par injection, comme l'injection SQL, peuvent être traitées de la même manière qu'un langage naturel.

Il est à savoir que ce module possède une grande similarité avec celui de compilation.
Les deux partagent presque le même pipeline : analyse lexicale, analyse syntaxique et analyse sémantique.
Ils ont des techniques en commun ; mais, celles du module compilation ne sont pas toujours applicables à un langage naturel.
Ce dernier est différent d'un langage de programmation qui est bien défini et donc contient moins d'ambiguïté.



\section*{Organisation générale du document}

Le traitement automatique d'un langage naturel passe par plusieurs niveaux : lexique, syntaxe, sémantique, pragmatique et discours. 
Les chapitres de ce document sont organisés par l'ordre de ces niveaux.
Chaque chapitre décrit une tâche élémentaire qui est utilisée pour construire des applications comme celles décrites dans le dernier chapitre.
Dans cette section, nous allons décrire chaque chapitre à part en précisant sa motivation.
Chaque chapitre est terminé par une discussion et quelques ressources (voir la section suivante).


Le premier chapitre a comme rôle d'introduire le domaine afin de savoir pourquoi nous devons apprendre ce domaine.
Nous commençons par une petite histoire sur ce domaine et celui de l'intelligence artificielle.
Ensuite, nous présentons les différents niveaux de traitement d'un langage naturel en détail.
Après, nous listons les différentes applications de ce domaine que ce soit les tâches élémentaires, les systèmes ou l'utilisation dans le secteur socio-économique comme dans l'éducation, la santé, etc. 


Le deuxième chapitre s'occupe de la tâche de prétraitement.
C'est une tâche nécessaire pour accomplir les autres tâches du traitement d'un texte.
Elle peut être vue comme l'étape d'analyse lexicale en compilation.
Son but est de préparer des unités de traitement de texte (dans notre cas : des mots).
Afin de traiter un texte il faut le diviser en phrases qui doivent être divisées en mots.
Quelques mots peuvent ne pas être utiles pour notre traitement ; donc, nous pouvons les supprimer.
D'autres mots peuvent avoir des variations morphologiques qui ne sont pas nécessaires pour notre application ; donc, nous pouvons garder que leurs radicaux.


Dans le troisième chapitre, nous introduisons le concept des modèles de langage.
Ce dernier essaye de trouver une distribution probabiliste sur les séquences des mots d'un langage.
Donc, étant donné une séquence des mots, nous pouvons estimer la probabilité d'occurrence d'un autre mot.
Parmi les applications de cette tâche, nous pouvons citer : l'autocomplétion, la détection des fautes d'orthographe et de syntaxe, etc.
Cette tâche peut être vue comme la version statistique d'analyse syntaxique, mais qui opère dans le niveau lexical.


Le quatrième chapitre s'intéresse par l'étiquetage des séquences : étant donné une séquence des mots, on veut attribuer une catégorie à chaque mot ou groupe de mots.
La tâche la plus connue dans ce cas est l'étiquetage morpho-syntaxique où nous cherchons la catégorie grammaticale de chaque mot.
Cette tâche peut être utilisée comme tâche préliminaire de l'analyse syntaxique.
Elle peut aussi être utilisée comme prétraitement des applications où la catégorie grammaticale d'un mot est importante.


L'analyse syntaxique sert à trouver la structure syntaxique d'une phrase.
Cela nous aide à déterminer si une phrase est grammaticalement correcte vis-à-vis une langue.
Dans le chapitre cinq, nous exploitons deux structure différentes : la structure constituante et la structure fonctionnelle.
Nous présentons des algorithme pour analyser une phrase donnée en se basant sur chaque structure.
Le but final de cette analyse est d'avoir un arbre syntaxique dans le cas où l'analyse est réussite.


Le sixième chapitre est réservé au sens des mots étant donnée qu'ils sont considérés comme l'unité principale de traitement.
Encoder les mots avec un vecteur OneHot a un sens très restreint : le mot existe ou non.
Afin d'encoder les mots, nous nous présentons deux types de sémantique : le sémantique lexicale et la sémantique statistique.
La première vise à encoder un mot sous forme d'un graphe en se basant sur ses relations sémantiques avec d'autres mots.
La deuxième utilise des méthodes statistiques afin d'encoder un mot sous forme d'un vecteur en se basant sur ses relations de co-occurrence avec d'autres mots.


Nous ne pouvons pas parler du sens des mots sans parler du sens des phrases présenté en chapitre huit.
Nous parlons un peu des rôles sémantiques qu'un groupe de mots puisse jouer dans une phrase.
Ensuite, nous présentons quelques méthodes pour l'étiquetage de ces rôles.
Après, nous discutons la représentation logique des phrases et comment l'obtenir en utilisant l'analyse sémantique.
La motivation dernière cette tâche est plus que claire : comprendre une phrase par la machine.


Le neuvième chapitre traite le sujet de la coréférence.
Cette dernière sert à trouver les relations de mentions entre des parties du texte.
Par exemple, nous essayons de lier les pronoms personnels avec l'entité référencée.
En plus de cette tâche, nous essayons d'introduire d'autres tâches similaires comme l'annotation sémantique et la reconnaissance des entités nommées.


Le dixième chapitre, étant le dernier, représente la somme de tous les chapitres qui le précèdent.
Il présente des différentes applications qui peuvent utiliser les tâches présentées précédemment.
Nous avons choisi des applications connues et présentes dans nos vies comme la traduction automatique.
Cela permet à l'étudiant de bien comprendre comment l'application est conçue vu qu'il comprenne comment l'utiliser.


\section*{Ressources supplémentaires}

La fin des chapitres contient des ressources supplémentaires.
Dans notre cas, ces ressources sont divisées en quatre catégories : exercices, tutoriels, TPs et Labs.
Les trois derniers sont des ressources externes qu'on puisse accéder via le répertoire Github du cours (\url{https://github.com/projeduc/ESI_2CS_TALN}).
Nous fournissons une petite description à ces ressources, mais pour avoir l'énoncé complet il faut les télécharger à part.
Ces quatre ressources peuvent ne pas être présents dans un chapitre.

%Exercices
Les exercices peuvent être des questions de réflexion ou des applications (calculs numérique, exécution d'une méthode manuellement).
Leur but est de mieux comprendre les différentes concepts présentées dans le chapitre.
%Tutoriels
Les tutoriels sont des petits programmes qui démontrent comment utiliser un outil.
Nous pouvons les considérer comme étant une documentation d'un outil avec application.
Leur but est de présenter les différents outils existants (et connus) pour la tâche en question.
%Travaux pratiques
Les travaux pratiques ont comme but d'apprendre un concept en le programmant à partir de zéro (from scratch).
Ici, nous ne devons pas utiliser des APIs ; il faut utiliser seulement les instructions basiques du langage de programmation.
L'étudiant ne doit pas programmer le tout ; seulement des parties omises du code.
En plus d'apprendre les concepts, les TPs présentent quelques méthodes d'évaluation.
%Laboratoires
Les laboratoires, comme les TPs, sont destinés à résoudre des problèmes avec machine.
La différence est que les laboratoires sont guidés et se basent sur les APIs présentés dans les tutoriels.
Leur but est que l'étudiant apprenne à utiliser les APIs pour résoudre un problème réel.


%=====================================================================
\ifx\wholebook\relax\else
% \cleardoublepage
% \bibliographystyle{../use/ESIbib}
% \bibliography{../bib/RATstat}
	\end{document}
\fi
%=====================================================================
