% !TEX TS-program = xelatex
% !TeX program = xelatex
% !TEX encoding = UTF-8
% !TEX spellcheck = fr

%=====================================================================
\ifx\wholebook\relax\else
	\documentclass{KodeBook}
	% !TEX TS-program = xelatex
% !TeX program = xelatex
% !TEX encoding = UTF-8
% !TEX spellcheck = fr

%\usepackage[T1]{fontenc}

%\usepackage[pdftex]{graphicx}

%\usepackage{listingsutf8}
%\usepackage{xcolor}
%\usepackage{times}
\usepackage{array}
\usepackage{natbib}
\usepackage{lscape}%to flip tables in a page
\usepackage{pdflscape}
\usepackage{longtable}
\usepackage{tabu}
\usepackage{wrapfig}
\usepackage{colortbl}
\usepackage{alltt}
\usepackage[french,lined]{algorithm2e}

\renewcommand{\cite}[1]{\citep{#1}}

%\usepackage[english]{babel}

\bibliographystyle{engdnat}%unsrtnat, plainnat

%\usepackage{pgf-umlcd}





\hypersetup{
	pdfkeywords={TALN; TAL; langue},
	pdfsubject={intelligence artificielle; traitement automatique de langages naturels}
}

\renewcommand{\UrlFont}{\ttfamily\footnotesize}

\DeclareAcronym{taln}{
	short = TALN ,
	long  = traitement automatique de langages naturels,
	class = abbrev
}

\DeclareAcronym{tal}{
	short = TAL ,
	long  = traitement automatique des langues,
	class = abbrev
}

\DeclareAcronym{ia}{
	short = IA ,
	long  = intelligence artificielle,
	class = abbrev
}

\DeclareAcronym{ibm}{
	short = IA ,
	long  = international business machines,
	class = abbrev
}

\DeclareAcronym{darpa}{
	short = DARPA ,
	long  = Defense Advanced Research Projects Agency,
	class = abbrev
}

\DeclareAcronym{ipa}{
	short = IPA ,
	long  = intelligent personal assistant,
	class = abbrev
}

\DeclareAcronym{iva}{
	short = IVA ,
	long  = intelligent virtual assistant,
	class = abbrev
}

\DeclareAcronym{ipa2}{
	short = IPA ,
	long  =  International Phonetic Alphabet,
	class = abbrev
}

%\makeglossaries

%\newacronym{oop}{OOP}{Object-oriented programming} 

	\begin{document}
		\mainmatter
	
\fi
%=====================================================================
\changegraphpath{../img/coref/}
\chapter{Détection de la coréférence}

\begin{introduction}
	\lettrine{L}{e} 
\end{introduction} 

\begin{exampleblock}{Exemple d'une phrase en français}
	\begin{center}
		\Large\bfseries
		%		\begin{enumerate}
		%			\item Mon chat a attrapé une souris avec ses griffes 
		%			\item Mon chat a attrapé une souris avec sa queue
		%			\item Mon chat a attrapé une souris avec un autre chat
		%		\end{enumerate}
		La fille a cueilli une fleur. Elle l'a sentie. Elle a une très bonne odeur.
	\end{center}
\end{exampleblock}

\begin{itemize}
	\item Qui a senti l'autre : la fille ou la fleur ?
	\item Qui a une très bonne odeur : la fille ou la fleur ?
\end{itemize}

%===================================================================================
\section{Références}
%===================================================================================

\subsection{Formes des références}

\begin{itemize}
	\item \optword{Pronoms}
	\begin{itemize}
		\item Personnels : \expword{\underline{Karim} est entré. \underline{Il} a commencé le cours.}
		\item Possessifs : \expword{\underline{Karim} a commencé \underline{son} cours.}
		\item ...
	\end{itemize}
	
	\item \optword{Syntagmes nominaux}
	\begin{itemize}
		\item \expword{J'ai un petit \underline{chat}. \underline{Cet animal} est très méchant.}
	\end{itemize}
	
	\item \optword{Noms propres} 
	\begin{itemize}
		\item \expword{L'\underline{école nationale supérieure d'informatique} se situe à Alger. Comme toutes les unniversités algériennes, il faut avoir le BAC pour étudier à l'\underline{ESI}.}
	\end{itemize}
	
	\item \optword{Anaphore zéro}
	\begin{itemize}
		\item \expword{\underline{Karim} a présenté et \underline{$ \phi $} expliqué le cours.}
		\item \expword{\underline{カリムさん}はESIに生きます。\underline{$ \phi $} あそこに教えます。}
	\end{itemize}
\end{itemize}

\subsection{Manière de référencement}

\begin{itemize}
	\item \optword{Anaphore}
	\begin{itemize}
		\item une référence vers un mot ou un syntagme précédent appelé \keyword{antécédent}.
		\item Ex. \expword{\underline{Le cours} est très long. \underline{Il} prendra plus de temps.}
	\end{itemize}
	
	\item \optword{Cataphore}
	\begin{itemize}
		\item une référence vers un mot ou un syntagme suivant appelé \keyword{postcédent}.
		\item Ex. \expword{\underline{Il} est très long, \underline{ce cours}!}
	\end{itemize}
	
	\item \optword{Antécédents partagés}
	\begin{itemize}
		\item une référence vers plusieurs mots rt/ou syntagmes
		\item Ex. \expword{\underline{Le cours} sera suivi par \underline{un exercice}. \underline{Ils} sont importants pour la compréhension.}
	\end{itemize}
	
	\item \optword{Syntagmes nominaux en coréférence}
	\begin{itemize}
		\item deux syntagmes nominaux où chacun est une référence vers l'autre
		\item Ex. \expword{\underline{Certains de nos collègues} nous ont vraiment soutenu. \underline{Ce genre de personnes} gagne notre gratitude.}
	\end{itemize}
\end{itemize}

\subsection{Propriétés des relations de coréférence}

\begin{itemize}
	\item \optword{Nombre} : Singulier, Duel, Pluriel 
	\begin{itemize}
		\item \expword{\underline{Le cours} sera suivi par \underline{un exercice}. \underline{Ils} sont importants pour la compréhension.}
	\end{itemize}
	
	\item \optword{Personne} : Première, Deuxième, Troisième
	\begin{itemize}
		\item \expword{\underline{Mon frère}\textsubscript{1} et \underline{moi}\textsubscript{2} avons réparé \underline{son vélo}\textsubscript{1} après \underline{le mien}\textsubscript{2}.}
	\end{itemize}
	
	\item \optword{Genre} : Masculin, Féminin, Neutre
	\begin{itemize}
		\item \expword{Lorsque la fille a rencontré son \underline{père}, \underline{il} a été content.}
	\end{itemize}
	
	\item \optword{Contraintes de théorie contraignante} : contraintes syntaxiques sur la relation mention-antécédent.
	\begin{itemize}
		\item \expword{Janet bought herself a bottle of fish sauce.} [herself = Janet]
		\item \expword{Janet bought het a bottle of fish sauce.} [her $\ne$ Janet]
	\end{itemize}

	\item \optword{Récence} : les entités mentionnées plus récemment sont plus probables à être un antécédent
	\begin{itemize}
		\item \expword{Le médecin a trouvé une vieille carte. Jim a trouvé \underline{une carte} encore plus ancienne. \underline{Elle} décrivait une île.}
	\end{itemize}
	
	\item \optword{Rôle grammatical} : les sujets sont plus probables que les objets
	\begin{itemize}
		\item \expword{\underline{Karim} est allé au restaurant avec son ami. \underline{Il} a demandé un plat de couscous.}
	\end{itemize}
	
	\item \optword{Sémantique du verbe} : la mention suit l'emphase du verbe (celui qui a causé l'évènement)
	\begin{itemize}
		\item \expword{\underline{John} telephoned Bill. \underline{He} lost the laptop.}
		\item \expword{John criticized \underline{Bill}. \underline{He} lost the laptop.}
	\end{itemize}
	
\end{itemize}

%===================================================================================
\section{Résolution des coréférences}
%===================================================================================

\hgraphpage{coref-arch.pdf}

\begin{itemize}
	\item \optword{Modèles de liaison}
	\begin{itemize}
		\item Mention-Pair : deux à deux
		\item Mention-Rank : un antécédent parmi plusieurs
		\item Entity-based : détecter des clusters de coréférences
	\end{itemize}
	
	\item \optword{Architecture de coréférences}
	\begin{itemize}
		\item Règles
		\item Caractéristiques définies manuellement
		\item Réseaux de neurones : embeddings
	\end{itemize}
\end{itemize}

\subsection{Détection de mention}

\begin{itemize}
	\item \optword{Extraction des entités}
	\begin{itemize}
		\item Syntagmes nominaux, pronoms, entités nommées
	\end{itemize}
	
	\item \optword{Filtrage des entités}
	\begin{itemize}
		\item Par règles linguistiques. Ex. Dans ``\expword{It is thought that ...}" le pronom ``\expword{It}" n'est pas une référence. Ici, on peut utiliser une liste des verbes cognitifs \expword{believe, think, etc.}
		\item Par apprentissage automatique : un détecteur des anaphores et un autre pour les antécédents
	\end{itemize}
	
	\item \optword{Formation des mentions}
	\begin{itemize}
		\item Regrouper les entités probables ensembles
	\end{itemize}
\end{itemize}

\subsection{Liaison}

Modèles Mention-Pair
\begin{itemize}
	\item Détecter s'il y a une coréférence entre deux pairs d'entités
	\item Classifieur binaire (coréférence, non coréférence)
	\item \textbf{Problème} : contradictions. Ex. \expword{Ms Kennedy $ \leftarrow $ Kennedy, Kennedy $ \leftarrow $ He}
\end{itemize}
\begin{figure}
	\centering
	\hgraphpage[.8\textwidth]{mention-pair-exp_.pdf}
	\caption{Exemple d'un modèle Mention-Pair \cite{2019-jurafsky-martin}}
\end{figure}

Modèles Mention-Rank
\begin{itemize}
	\item Parmi les mentions précédentes d'une entité, décider quel est l'antécédent
	\item $ \hat{a} = \arg\max\limits_{a_i \in \{\epsilon, a_1, \ldots, a_{n-1}\}} P(a_i|a_n) $
	\item $ \epsilon $ : veut dire, il n'y a pas d'antécédent
\end{itemize}
\begin{figure}
	\centering
	\hgraphpage{mention-rank-exp_.pdf}
	\caption{Exemple d'un modèle Mention-Rank \cite{2019-jurafsky-martin}}
\end{figure}

Modèles Entity-based
\begin{itemize}
	\item Appelés aussi : modèles cluster-ranking
	\item \textbf{Entrée} : deux clusters des mentions
	\item Vérification si les mentions sont compatibles
	\item Estimation si un cluster est un antécédent de l'autre comme dans Mention-Rank
	\item Si les cluster représentent les mêmes mentions, on les fusionne
	\item \textbf{Sortie} : Un ensemble des clusters
\end{itemize}

\subsection{Évaluation}

\begin{itemize}
	\item \keyword{MUC} : \optword{Message Understanding Conference}
	\begin{itemize}
		\item Métrique basée sur les liens (link-based metric)
		\item Nombre des liens binaires communs entre la référence et le système
	\end{itemize}
	\item \keyword{B\textsuperscript{3}}
	\begin{itemize}
		\item Métrique basée sur les mentions (mention-based metric)
		\item R/P globale est calculé en fonction de R/P des mentions individuelles
	\end{itemize}
	\item \keyword{CEAF} : \optword{Constrained entity-alignment F-Measure}
	\begin{itemize}
		\item mention-based OU entity-based (2 versions selon la similarité utilisée)
		\item Une mention du système est alignée avec une seule de référence
	\end{itemize}
	\item \keyword{BLANC} 
	\begin{itemize}
		\item Métrique basée sur les liens (link-based metric)
		\item R/P globale est la moyenne de R/P des liens de coréférences et ceux des non coréférences
	\end{itemize}
	\item \keyword{LEA} : \optword{Link based entity aware}
	\begin{itemize}
		\item La taille de l'entité comme mesure d'importance
		\item Les liaison de coréférence résolues sont évaluées
	\end{itemize}
\end{itemize}

%===================================================================================
\section{Tâches connexes}
%===================================================================================

\begin{itemize}
	\item Tâches utilisées dans la détection de la coréférence
	\begin{itemize}
		\item Tâches de prétraitement (chapitre 2)
		\item Étiquetage morpho-syntaxique (chapitre 4)
		\item Reconnaissance des entités nommées
	\end{itemize}
	\item Tâches similaires à la détection de la coréférence
	\begin{itemize}
		\item Annotation sémantique (Entity linking)
		\item Attribution de la citation : trouver qui a dit/écrit un discours
	\end{itemize}
\end{itemize}

\subsection{Annotation sémantique (Entity linking)}

\begin{itemize}
	\item Associer une mention dans un texte à une représentation d'une entité dans une base de connaissances structurée
	\item utile pour avoir une connaissance approfondie sur les entités du texte dans le monde réel
	\item \keyword{Wikification} : relier une mention à une page de Wikipédia
	\item Les étapes de l'annotation sémantique : 
	\begin{itemize}
		\item \optword{Détection de la mention} : détecter l'ensemble des entités d'un base de connaissance liées à une mention en utilisant des requêtes.
		\item \optword{Désambigüisation de la mention} : trouver l'entité la plus probable en utilisant l'apprentissage automatique
	\end{itemize}
\end{itemize}

\subsection{Reconnaissance des entités nommées}

\begin{itemize}
	\item localiser et classer les entités nommées dans un texte
	\item \keyword{entité nommée} : personnes, places, organisations, quantités, etc.
	\item une sous tâche de l'extraction des données
	\item \textbf{En anglais} : Named-entity recognition (NER)
	\item Techniques de reconnaissance 
	\begin{itemize}
		\item \optword{Règles} : utiliser des règles et des listes de noms pour chercher les entités et détecter leurs types
		\item \optword{Apprentissage avec caractéristiques} : embeddins du mot et ses voisins, les préfixes et les suffixes, l'appartenance à une liste, etc.
		\item \optword{Étiquetage de séquences} : classer les mots en entités en les traitant comme une séquence. Une entité peut avoir plusieurs mots qui commence par \keyword{B} (Begin) suivi par \keyword{I} (Inside). Les mots sans classe ont le tag \keyword{O} (Outside). 
		
		\expword{\scriptsize
			$ \underbrace{Google}_{B-ORG} $ 
			$ \underbrace{LLC}_{I-ORG} $ 
			$ \underbrace{est}_{O} $ 
			$ \underbrace{fond\text{\textit{é}}e}_{O} $ 
			$ \underbrace{dans}_{O} $ 
			$ \underbrace{la}_{O} $ 
			$ \underbrace{Silicon}_{B-LOC} $ 
			$ \underbrace{Valley}_{B-LOC} $ 
			$ \underbrace{par}_{O} $ 
			$ \underbrace{Larry}_{B-PER} $ 
			$ \underbrace{Page}_{I-PER} $ 
			$ \underbrace{et}_{O} $ 
			$ \underbrace{Sergey}_{B-PER} $ 
			$ \underbrace{Brin}_{I-PER} $
		}
	\end{itemize}
\end{itemize}



\begin{discussion}



\end{discussion}

%=====================================================================
\ifx\wholebook\relax\else
% \cleardoublepage
% \bibliographystyle{../use/ESIbib}
% \bibliography{../bib/RATstat}
	\end{document}
\fi
%=====================================================================
