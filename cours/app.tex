% !TEX TS-program = xelatex
% !TeX program = xelatex
% !TEX encoding = UTF-8
% !TEX spellcheck = fr

%=====================================================================
\ifx\wholebook\relax\else
	\documentclass{KodeBook}
	% !TEX TS-program = xelatex
% !TeX program = xelatex
% !TEX encoding = UTF-8
% !TEX spellcheck = fr

%\usepackage[T1]{fontenc}

%\usepackage[pdftex]{graphicx}

%\usepackage{listingsutf8}
%\usepackage{xcolor}
%\usepackage{times}
\usepackage{array}
\usepackage{natbib}
\usepackage{lscape}%to flip tables in a page
\usepackage{pdflscape}
\usepackage{longtable}
\usepackage{tabu}
\usepackage{wrapfig}
\usepackage{colortbl}
\usepackage{alltt}
\usepackage[french,lined]{algorithm2e}

\renewcommand{\cite}[1]{\citep{#1}}

%\usepackage[english]{babel}

\bibliographystyle{engdnat}%unsrtnat, plainnat

%\usepackage{pgf-umlcd}





\hypersetup{
	pdfkeywords={TALN; TAL; langue},
	pdfsubject={intelligence artificielle; traitement automatique de langages naturels}
}

\renewcommand{\UrlFont}{\ttfamily\footnotesize}

\DeclareAcronym{taln}{
	short = TALN ,
	long  = traitement automatique de langages naturels,
	class = abbrev
}

\DeclareAcronym{tal}{
	short = TAL ,
	long  = traitement automatique des langues,
	class = abbrev
}

\DeclareAcronym{ia}{
	short = IA ,
	long  = intelligence artificielle,
	class = abbrev
}

\DeclareAcronym{ibm}{
	short = IA ,
	long  = international business machines,
	class = abbrev
}

\DeclareAcronym{darpa}{
	short = DARPA ,
	long  = Defense Advanced Research Projects Agency,
	class = abbrev
}

\DeclareAcronym{ipa}{
	short = IPA ,
	long  = intelligent personal assistant,
	class = abbrev
}

\DeclareAcronym{iva}{
	short = IVA ,
	long  = intelligent virtual assistant,
	class = abbrev
}

\DeclareAcronym{ipa2}{
	short = IPA ,
	long  =  International Phonetic Alphabet,
	class = abbrev
}

%\makeglossaries

%\newacronym{oop}{OOP}{Object-oriented programming} 

	\begin{document}
		\mainmatter
	
\fi
%=====================================================================
\changegraphpath{../img/app/}
\chapter{Quelques applications}

\begin{introduction}[LES LANGUE\textcolor{white}{S}]
	\lettrine{S}{elon} l'interaction avec l'utilisateur, un système peut être interactif ou non.
	En se basant sur la sortie, un système peut générer un ensemble de classes ou un autre texte. 
	Un système de traitement du langage naturel peut traiter de la parole ou du texte. 
	En utilisant ces critères, on peut classifier les applications du TALN en quatre catégories : Transformation, Interaction, Classification et Parole. 
	La transformation sert à prendre un texte en entrée et générer un autre en sortie ; comme la traduction automatique et le résumé automatique. 
	L'interaction comporte les applications qui interagissent avec l'utilisateur ; comme les systèmes de question/réponse et de dialogue.
	La classification prend un texte en entrée et infère un classe ; comme l'analyse des sentiments et la lisibilité.
	Enfin la parole consiste de la reconnaissance et la synthèse de la parole.
	Dans ce chapitre, on va présenter deux applications par catégorie.
\end{introduction} 

Chaque jour beaucoup d'informations sont générées ; en général, sous forme de textes non structurées. 
L'anglais est la langue la plus répondue ; mais, on peut remarquer une augmentation du contenu des autres langues.
Traiter ces textes sera bénéfique dans la résolution de nous besoins.
Reprenons quelques motivations mentionnées dans le premier chapitre :
\begin{itemize}
	\item Commerce : publicité, service clientèle, intelligence de marché, recrutement, etc.
	\item E-Gouvernance : communication gouvernement/citoyens, fouille d'opinions, etc.
	\item Santé : rechercher, analyser, interpréter et structurer les documents médicaux, prédire les maladies, utiliser les assistants virtuels, etc.
	\item Éducation :  évaluation de la langue, correction des erreurs, apprentissage en ligne, etc.
\end{itemize}

%===================================================================================
\section{Traduction automatique}
%===================================================================================

La traduction automatique consiste à transformer un texte écrit en une langue origine vers un texte écrit en une langue destinataire.
La motivation de cette tâche est claire : casser le barrière entre les être humains en facilitant la communication. 
Plusieurs méthodes ont été proposées pour cette tâche en utilisant des différentes techniques. 
En se basant sur le type de ces dernières, les méthodes de traduction automatique peuvent être classifiées comme indiqué dans la figure \ref{fig:mt-class}. 
L'approche directe est la plus simple ; elle consiste à une traduction mot par mot. 
Donc, elle est utile seulement pour les langues qui sont proches syntaxiquement.
On peut citer deux approches à base de règles : par transfert et en utilisant une interlingua.
Les méthodes par transfert se basent sur des règles pour transformer un arbre syntaxique de la langue origine vers un arbre syntaxique de la langue destinataire. 
L'autre idée est de proposer toute une langue qui doit être universelle.
Ensuite, on peut concevoir des encodeurs à partir des langages naturels vers cette langue et des décodeurs de cette langue vers des langages naturels.
On peut utiliser des corpus pour entraîner un système d'apprentissage automatique à traduire d'une langue vers une autre. 
Donc, on doit se baser des statistiques. 
La terminologie ``statistique" est souvent utilisée pour indiquer qu'il s'agit de l'apprentissage bayésien. 
Ces méthodes ont besoins de beaucoup de données pour entraîner les probabilités de chaque mot. 
En plus, des parties de textes sont toujours utilisées telles qu'elles sont. 
Pour améliorer la classification, on peut utiliser ces parties comme unités ; d'où les méthodes à base d'exemples.
Finalement, les méthodes les plus répondues ces jours-là sont à base des réseaux de neurones. 
Le problème de traduction automatique peut être représenté comme un encodeur/décodeur.

\begin{figure}[!ht]
	\centering
	\hgraphpage[.8\textwidth]{translation-classif_noir.pdf}
	\caption{Approches de la traduction automatique}
	\label{fig:mt-class}
\end{figure}

\subsection{Approche directe}

Dans l'approche directe, on essaye de traduire le texte mot par mot. 
Le texte source $S$ est traité comme une série de mots. 
Chaque mot $S_i$ est remplacé par un mot $T_i$ dans le texte destinataire $T$ en utilisant un dictionnaire bilingue. 
Donc, pour implémenter une telle méthode, il nous faut un outil d'analyse morphologiques de la langue source et un dictionnaire bilingue bien conçu. 
Les langues (source et destinataire) doivent être proches syntaxiquement (structures grammaticales proches) puisque cette approche ne supporte que le niveau morphologique. 
La traduction mot à mot est suivie par une étape de post-traitement pour organiser l'ordre des mots. 
Exemple, \expword{SVO \textrightarrow VSO}, \expword{adj + N \textrightarrow N + Adj}
Les systèmes qui suivent cette approche sont ceux développés avant 1967 : Météo, Weidner, CULT et Systran (premières versions).

\subsection{Approche par transfert}

Traduire les textes mot à mot limite le nombre des tuples de langues (source, destinataire) qu'on puisse traiter. 
L'idée de cette approche est de trouver l'arbre syntaxique de la phrase source $S$ en utilisant l'analyse syntaxique. 
Ensuite, on cherche la traduction des mots de la langue source $S_i$ dans la langue destinataire $T_i$ en utilisant un dictionnaire bilingue.
Après, on applique des règles pour transformer l'arbre syntaxique de la langue source vers une arbre syntaxique de la langue destinataire (comme l'exemple de la figure \ref{fig:mt-transfert-exp}).
Ces règles sont définies manuellement ou inférées en utilisant l'apprentissage automatique.
Finalement, on génère le texte traduit $T$ à partir de l'arbre syntaxique final.

\begin{figure}[!ht]
	\centering
	\hgraphpage[.55\textwidth]{MT-tranfert-exp_.pdf}
	\caption{Un exemple de règles de transfert syntaxique \cite{06-quah}}
	\label{fig:mt-transfert-exp}
\end{figure}

Un des systèmes qui se basent sur cette approche est Apertium\footnote{Apertium : \url{https://www.apertium.org/}} \cite{11-forcada-al}.
La figure \ref{fig:apertium-arch} représente l'architecture de ce système qui ce compose des modules suivants :
\begin{itemize}
	\item \textit{deformatter} : éliminer les informations de format et garder seulement le texte. 
	Il est utiliser pour lire le texte en utilisant plusieurs formats comme XML.
	\item \textit{morphological analyser} : segmentation du texte et attribution à chaque mot une liste des lemmes avec les catégories grammaticales possibles.
	\item \textit{PoS tagger} : un outil d'étiquetage morpho-syntaxique pour avoir les catégories grammaticales des mots.
	\item \textit{lexical transfer} : traduire les mots (ou multi-mots) de la langue source vers la langue destinataire en utilisant un dictionnaire bilingue.
	\item \textit{structural transfer} : c'est un module qui fournit le transfert syntaxique. 
	Il contient un sous-module \textit{chunker} qui applique une analyse syntaxique de surface (Chunking) sur le texte de source. 
	Le sous-module \textit{interchunk} applique des règles de transfert de la structure source vers la structure destinataire. 
	Le sous-module \textit{postchunk} applique des opérations de post-traitement sur la structure destinataire.
	\item \textit{morphological generator} : appliquer des transformations morphologiques comme la conjugaison sur les mots. 
	\item \textit{post-generator} : appliquer des opérations d'orthographe. 
	Par exemple, en anglais \expword{a + institute = an institute}.
	\item \textit{reformatter} : formater la réponse en utilisant un format spécifique comme XML, JSON, etc.
\end{itemize}

\begin{figure}[!ht]
	\centering
	\hgraphpage[.85\textwidth]{apertium-arch_.pdf}
	\caption{Architecture du système Apertium \cite{11-forcada-al}}
	\label{fig:apertium-arch}
\end{figure}

\subsection{Approche interlingue}

Dans la traduction automatique, on doit implémenter un système pour chaque pair de langues de source et de destination. 
Par exemple, si on possède un système de traduction du français vers l'anglais et un autre de l'anglais vers l'arabe, on peut utiliser les deux systèmes afin de traduire du français vers l'arabe. 
Dans ce cas, l'anglais a été utilisée comme langue intermédiaire. 
L'idée de l'approche interlingua est d'utiliser un langage intermédiaire comme indiqué dans la figure \ref{fig:mt-interlangue}.
Pour ajouter une langue source, on doit seulement implémenter un analyseur de cette langue vers l'interlingua. 
Comme ça, cette langue peut être traduite vers toutes les langues disponibles comme langues destinataires du système. 
Afin d'ajouter une nouvelle langue destinataire, il suffit d'implémenter un générateur de cette langue à partir de l'interlingua. 
Donc, la traduction passe par les étapes suivantes :
\begin{itemize}
	\item Analyser le texte source $S$ pour avoir un arbre syntaxique
	\item Utiliser un dictionnaire entre le langage source et les concepts de l'interlingua 
	\item Transformer l'arbre syntaxique (langue source) vers l'interlingue
	\item Transformer l'interlingue vers une arbre syntaxique (langue destinataire)
	\item Utiliser un dictionnaire entre le langage destinataire et les concepts de l'interlingua 
	\item Générer le texte destinataire $T$
\end{itemize}

\begin{figure}[!ht]
	\centering
	\hgraphpage[.7\textwidth]{MT-Interlingua_noir.pdf}
	\caption{Motivation de l'approche interlingue}
	\label{fig:mt-interlangue}
\end{figure}

L'interlingua doit être une représentation abstraite indépendante des langues ; elle doit être un langage universel.
KANT \cite{98-czuba-al} est un exemple d'une interlingua (voir la figure \ref{fig:kant-exp}).

\begin{figure}[!ht]
	\centering
\begin{multicols}{2}
\bfseries\tiny
\begin{verbatim}
(*A-REMAIN  ; action rep for 'remain'
    (FORM FINITE)
    (TENSE PAST)
    (MOOD DECLARATIVE)
    (PUNCTUATION PERIOD)
    (IMPERSONAL -) ; passive + expletive subject
    (ARGUMENT-CLASS THEME+PREDICATE) ; predicate argument structure
    (Q-MODIFIER ; PP semrole (generic)
        (*K-DURING ; PP interlingua
            (POSITION FINAL) ; clue for translation
 		    (OBJECT ; PP object semrole
 		        (*O-TIME ; object rep for 'time'
 		            (UNIT -)
 		            (NUMBER SINGULAR)
 		            (REFERENCE DEFINITE)
 		            (DISTANCE NEAR)
 		            (PERSON THIRD)))))
    (THEME ; object semrole
        (*O-DEFAULT-RATE ; object rep for ’default rate’
            (PERSON THIRD)
            (UNIT -)
            (NUMBER SINGULAR)
            (REFERENCE DEFINITE)))
    (PREDICATE ; adjective phrase semrole
        (*P-CLOSE ; property rep for ’closer’
            (DEGREE POSITIVE)
            (Q-MODIFIER
                (*K-TO
                    (OBJECT
                        (*O-ZERO
                        (UNIT -)
                        (NUMBER SINGULAR)
                        (REFERENCE NO-REFERENCE)
                        (PERSON THIRD))))))))
\end{verbatim}
\end{multicols}
	\caption{Représentation de la phrase "\textit{The default rate remained close to zero during this time.}" avec KANT interlingua \cite{98-czuba-al}}
	\label{fig:kant-exp}
\end{figure}

KANTOO \cite{00-nyberg-al} est un système de traduction automatique qui utilise KANT comme interlingua.
Son architecture est illustrée dans la figure \ref{fig:mt-kantoo-arch}.
Le système garde une base de connaissance contenant des règles manuelles pour la transformation entre les langages naturelles et KANT. 
L'analyseur applique une analyse syntaxique pour ensuite transférer la représentation sémantiquement vers KANT. 
Le générateur applique l'opération inverse : générer un texte à partir de la représentation KANT.

\begin{figure}[!ht]
	\centering
	\hgraphpage[\textwidth]{kantoo-arch_noir.pdf}
	\caption{Architecture du système KANTOO, traduit à partir de \cite{00-nyberg-al}}
	\label{fig:mt-kantoo-arch}
\end{figure}


\subsubsection{Approche statistique}

Étant donné un texte de source $S$ et un autre destinataire $T$, la probabilité que $T$ soit généré à partir de $S$ est estimée en utilisant la théorie de Bayes comme indiqué dans l'équation \ref{eq:mt-stat-nb}.
\begin{equation}\label{eq:mt-stat-nb} 
p(T|S) = \frac{p(T) p(S|T)}{p(S)} \propto \underbrace{p(T)}_\text{Cohérence} \underbrace{p(S|T)}_\text{Fidélité}
\end{equation}
Le problème de traduction automatique revient à trouver le texte destinataire $\hat{T}$ qui maximise cette probabilité comme indiqué dans l'équation \ref{eq:mt-stat-max}
\begin{equation}\label{eq:mt-stat-max} 
\hat{T} = \arg\max_{T} p(T) p(S|T)
\end{equation}
La probabilité $p(T)$ peut être calculée en utilisant un modèle de langage entrainé sur la langue destinataire. 
Supposant que le modèle de langage est un modèle \keyword[N]{N-gramme}, cette probabilité peut être calculée par l'équation \ref{eq:mt-stat-ngramme}
\begin{equation}\label{eq:mt-stat-ngramme} 
p(T) = \prod_{j=1}^m p(t_j|t_{j-N+1}\ldots t_{j-1})
\end{equation}

Nous avons vu comment entraîner un modèle de langage. 
Mais, afin d'entraîner le modèle de traduction $p(S|T)$, il faut avoir un corpus aligné : les mots de la langue source doivent être liés avec ceux correspondants de la langue destinataire.
Il faut mentionner que la taille pose un problème lors de l'alignement. 
On peut considérer cette probabilité comme la somme des probabilités de tous les alignements $A$ possibles, comme indiqué dans l'équation \ref{eq:mt-stat-align1}.
\begin{equation}\label{eq:mt-stat-align1}
p(S|T) = \sum_{A} p(S, A | T)
\end{equation}
%\begin{equation}\label{eq:mt-stat-align2a}
%A^* = \arg\max_A p(S, A | T)
%\end{equation}
Il faut mentionner qu'on doit avoir $(m + 1)^n$ alignements possibles pour un texte source $S$ de taille $n$ et un texte traduit $T$ de taille $m$.
La probabilité qu'un texte source $S$ soit généré avec un alignement $A$ sachant un texte destinataire $T$ peut être calculée en se basant sur les probabilités que chaque mot de $S$ soit aligné avec $A_i$ sachant tous les mots passés de $S$ ($S_1^{i-1} = S_1 \ldots S_{i-1}$), les alignements passés de $A$ et tous les mots du texte traduit. 
Cette probabilité est formulée dans l'équation \ref{eq:mt-stat-align2}.
%On peut multiplier cette probabilité par la probabilité de générer $m$ 
\begin{equation}\label{eq:mt-stat-align2}
p(S, A | T) = \prod_{i=1}^{n} p(S_i, A_i | S_1^{i-1}, A_1^{i-1}, T_1^{m})
\end{equation}
La probabilité élémentaire de $S_i$ et $A_i$ peut être décomposée selon l'équation \ref{eq:mt-stat-align4}.
\begin{equation}\label{eq:mt-stat-align4}
p(S_i, A_i | S_1^{i-1}, A_1^{i-1}, T_1^{m}) = p(A_i | S_1^{i-1}, A_1^{i-1}, T_1^{m}) p(S_i | S_1^{i-1}, A_1^{i}, T_1^{m})
\end{equation}

La probabilité $p(S_i | S_1^{i-1}, A_1^{i}, T_1^{m})$ peut être estimée en utilisant un corpus d'entrainement aligné. 
Quant à la probabilité $p(A_i | S_1^{i-1}, A_1^{i-1}, T_1^{m})$, il existe plusieurs modèles pour l'estimer comme les modèles d'IBM \cite{1993-brown-al}.
Le premier modèle d'IBM suppose une distribution uniforme entre les $m$ mots de $T$. 
Dans ce cas, cette probabilité sera calculée en utilisant l'équation \ref{eq:mt-stat-ibm1}. 
\begin{equation}\label{eq:mt-stat-ibm1}
p(A_i | S_1^{i-1}, A_1^{i-1}, T_1^{m}) = \frac{1}{m+1}
\end{equation}
Donc, l'équation \ref{eq:mt-stat-align2} peut être reformulée comme l'équation \ref{eq:mt-stat-ibm1est}.
\begin{equation}\label{eq:mt-stat-ibm1est}
p(S|T) = \frac{1}{(m+1)^n} \sum_{A} \prod_{i=1}^{n} p(S_i | S_1^{i-1}, A_1^{i}, T_1^{m})
\end{equation}
Dans le modèle IBM2, la probabilité des alignements se base seulement sur la position actuelle $i$, le mot actuel $A_i$, la taille du texte source $n$ et la taille du texte destinataire $m$. 
Cette probabilité est entrainée en comparant l'alignement juste avec le reste des alignements.
Elle peut être formulée comme l'équation \ref{eq:mt-stat-ibm2}.
\begin{equation}\label{eq:mt-stat-ibm2}
p(A_i | S_1^{i-1}, A_1^{i-1}, T_1^{m}) = p(A_i | i, n, m)
\end{equation}
Donc, l'équation \ref{eq:mt-stat-align2} peut être reformulée comme l'équation \ref{eq:mt-stat-ibm2est}.
\begin{equation}\label{eq:mt-stat-ibm2est}
p(S|T) = \sum_{A} \prod_{i=1}^{n} p(A_i | i, n, m) p(S_i | S_1^{i-1}, A_1^{i}, T_1^{m})
\end{equation}
Dans le modèle HMM \cite{96-vogel-al}, la probabilité de l'alignement d'un mot $A_i$ se base sur l'alignement du mot précédent $A_{i-1}$.
Cela peut être représenté par l'équation \ref{eq:mt-stat-hmm}.
\begin{equation}\label{eq:mt-stat-hmm}
p(A_i | S_1^{i-1}, A_1^{i-1}, T_1^{m}) = p(A_i | A_{i-1}, m)
\end{equation}
Donc, l'équation \ref{eq:mt-stat-align2} peut être reformulée comme l'équation \ref{eq:mt-stat-hmmest}.
\begin{equation}\label{eq:mt-stat-hmmest}
p(S|T) = \sum_{A} \prod_{i=1}^{n} p(A_i | A_{i-1}, m) p(S_i | S_1^{i-1}, A_1^{i}, T_1^{m})
\end{equation}

\subsubsection{Approche par exemples}

Des fois, on trouve des segments de la langue source qui sont toujours alignés avec d'autre en langue destinataire.
Donc, on peut calculer la probabilité d'un segment (ensemble de mots consécutifs) par rapport à un autre. 
Moses\footnote{Moses : \url{http://statmt.org/moses/}} \cite{07-koehn-al} est un système de traduction par exemples.
Il entraîne quatre modèles statistiques :
\begin{itemize}
	\item $\phi(S|T)$ : une table de traduction des segments  composée des segments $S$, segments $T$ équivalents et les probabilités.
	\item $LM$ : modèle de langage sur la langue destinataire 
	\item $ D(T, S) $ : modèle de distorsion qui attribue un coût à chaque réorganisation des segments d'une phrase  
	\item Pénalité de mots $W(T)$ : pour qu'une traduction ne soit pas longue ou courte
\end{itemize}
Ces modèles sont utilisés pour estimer la probabilité d'une traduction $T$ sachant un texte source $S$ avec des poids en suivant l'équation \ref{eq:mt-exemples}.
Pour estimer $\hat{T}$, \keyword[B]{Beam search} est utilisé.
\begin{equation}\label{eq:mt-exemples}
p(T|S) = \phi(S|T)^{poids_{\phi}} \times LM^{poids_{LM}} \times D(T, S)^{poids_{D}} \times W(T)^{poids_{W}}
\end{equation}

\subsubsection{Approche neuronale}

La traduction automatique est un problème d'encodage-décodage ; on encode le texte source vers une représentation partagée qui sera décodée vers un texte destinataire.
Donc, on peut utiliser un encodeur-décodeur basés sur un réseau de neurones récurrent ; il est appelé modèle sequence-to-sequence (seq2seq).
L'encodeur a comme but d'encoder une phrase du langage source $S$. Le résultat est une représentation du contexte sous forme d'un vecteur. 
Ce vecteur du contexte est décodé vers une phrase du langage destinataire $T$. 
Formellement, la probabilité de génération du texte $T$ sachant un texte $S$ est décomposée comme indiqué dans l'équation \ref{eq:mt-nn-prob}.
\begin{equation}\label{eq:mt-nn-prob}
p(T|S) = p(t_1|S) p(t_2|S, t_1) p(t_3|S, t_1, t_2)\ldots p(t_m|S, t_1\ldots t_{m-1})
\end{equation}
Donc, la solution du problème  revient à maximiser cette probabilité comme indiqué dans l'équation \ref{eq:mt-nn-probmax}.
\begin{equation}\label{eq:mt-nn-probmax}
\hat{T} = \arg\max_{T} \prod_{i=1}^{m} p(t_i | S, t_1\ldots t_{i-1})
\end{equation}
%Le mot destinataire $t_j$ est un mot $w$ qui appartient au vocabulaire du langage destinataire.
%Le mot $\hat{t}_j$ suivant est celui appartenant au vocabulaire $V$ de la langue destinataire et qui a la plus grande probabilité sachant le texte source $S$ et les mots générés avant lui. 
%Cela est indiqué dans l'équation \ref{eq:mt-nn-est}.
%\begin{equation}\label{eq:mt-nn-est}
%\hat{t}_j = \arg\max_{w \in V} p(w | S, t_1\ldots t_{j-1})
%\end{equation}

Un système de traduction automatique bien connu qui utilise l'approche neuronale est Google Translate\footnote{Google Translate : \url{https://translate.google.com/}} \cite{2016-wu-al}.
La figure \ref{fig:mt-google} représente l'architecture du système de traduction automatique neuronale de Google. 
L'encodeur se compose de huit couches LSTM dont le premier est un Bi-LSTM afin de capturer le contexte future. 
Le texte d'entrée $X = x_1 \ldots x_n$ sera encodé comme une séquence $\bar{x}_1 \ldots \bar{x}_n$. 
Le décodeur, lui aussi, se compose de 8 LSTM.
Soit $y_{i-1}$ la sortie passée de l'encodeur, la prochaine entrée est calculée en utilisant un mécanisme d'attention. 
On passe chaque vecteur $\bar{x}_t$ combiné avec $y_{i-1}$ par un réseau de neurone à propagation avant avec une seule couche cachée, appelé $AttentionFunction$ pour avoir un nouveau vecteur $s_t = AttentionFunction(y_{i-1}, \bar{x}_t)$. 
Une fois tous les $n$ vecteurs sont encodés, on applique une fonction Softmax sur eux.
Ensuite, le résultat $p_t$ est utilisé comme pondération du vecteur $\bar{x}_t$ dans une somme pondérée de tous les $n$ vecteurs comme indiqué dans l'équation \ref{eq:mt-google-next}
\begin{equation}\label{eq:mt-google-next}
a_i = \sum_{t=1}^{n} p_t \cdot \bar{x}_t \quad \text{ où }\quad p_t = \frac{e^{s_t}}{\sum_{j=1}^n e^{s_j}}
\end{equation}
Le vecteur résultat est utilisé comme entrée du décodeur pour générer le mot suivant $y_i$.
Afin de décoder la sortie, les auteurs utilisent \keyword[B]{Beam search}.

\begin{figure}[!ht]
	\centering
	\hgraphpage[.89\textwidth]{googlet_.pdf}
	\caption{Architecture du système de traduction automatique neuronale de Google \cite{2016-wu-al}}
	\label{fig:mt-google}
\end{figure}

Un autre système qui suit l'approche neuronale est OpenNMT\footnote{OpenNMT : \url{https://opennmt.net/}} \cite{17-klein-al} dont l'architecture est indiquée dans la figure \ref{fig:mt-opennmt}.
Il utilise un modèle seq2seq avec le mécanisme d'attention.

\begin{figure}[!ht]
	\centering
	\hgraphpage[.6\textwidth]{opennmt_.pdf}
	\caption{Architecture du système de traduction automatique neuronale OpenNMT \cite{17-klein-al}}
	\label{fig:mt-opennmt}
\end{figure}


%===================================================================================
\section{Résumé automatique}
%===================================================================================

Le résumé automatique consiste à transformer un texte (images, vidéos ou un son) d'une forme longue vers une forme réduite (plus concise).
Cette tâche est motivée par le gagne du temps de lecture et de traitement des grandes quantités d'informations.
Un système de classification peut être classifié en utilisant plusieurs critères ; il peut appartenir à plusieurs classes au même temps.
Ces critères sont regroupées en trois catégories : document d'entrée, le but et le document de sortie \cite{98-hovy-lin,99-sparckjones}.
Cette classification est représentée par la figure \ref{fig:ats-class}.

\begin{figure}[!ht]
	\centering
	\hgraphpage[.8\textwidth]{sum-classif_noir.pdf}
	\caption{Classification des méthodes de résumé automatique selon \cite{98-hovy-lin,99-sparckjones}}
	\label{fig:ats-class}
\end{figure}

Selon de document d'entrée, une méthode peut être classifiés en utilisant trois critères : l'unité, la spécialité et la forme. 
Un système de résumé automatique peut être mono-document (un document en entrée) ou multi-documents (plusieurs documents en entrée). 
Il peut être spécialisé à un domaine (Ex. \expword{médecine}) ou général (n'importe quel domaine).
La forme d'un document d'entrée comporte plusieurs critères : la structure (document structuré ou non), l'échelle (taille du document : un tweet, un livre, etc.), le médium (texte, audio, vidéo, image) ou le genre (nouvelles, interviews, romans, etc.).

Selon le but, une méthode peut être classifiés en utilisant trois critères : le public, la fonction et la situation. 
Un système de résumé automatique peut être par requête (utiliser une requête utilisateur afin de générer le résumé) ou générique (utiliser le sujet du document afin de générer le résumé).
Il peut être indicatif (une description globale du document) ou informatif (l'information essentielle du document). 
Il peut générer le fond (tout ce qu'est important dans le document) ou juste les nouvelles (tout ce qui est nouveau).

Selon le document de sortie, une méthode peut être classifiés en utilisant trois critères : la dérivation, la partialité et le format.
Un système de résumé automatique peut être extractif (extraire des unités comme les phrases pour avoir un résumé) ou abstractif (générer un nouveau texte avec des nouveaux mots).
Il peut être partiel (il n'ajoute aucun opinion au résumé) ou évaluatif (il ajoute des opinions au résumé).
Le format du résumé peut être fixe (la même structure des résumés) ou flottant (des structures différentes selon des paramètres utilisateur).

Selon le système de résumé voulu, on peut décider une approche. 
Il existe plusieurs classifications des approches, parmi ces classifications :
\begin{itemize}
	\item Classification de \cite{12-nenkova-mckeown} : les méthodes sont classifiées selon leur représentation en deux classes.
	Les méthodes qui se basent sur le sujet : mots du sujet, fréquences, analyse sémantique latente, 
	modèles de sujets bayésiens, clustering.
	Les méthodes qui utilisent des indicateurs : par graphes, apprentissage automatique.
	\item Classification de \cite{12-lloret-palomar} : les méthodes sont classifiées selon les techniques utilisées.
	Elles peuvent être : statistique, par graphes, basée discours ou par apprentissage automatique.
	\item Classification de \cite{19-aries-al} : elle est similaire à la classification passée, mais elle regroupe les méthodes selon l'utilisation des ressources (puissance de calcul et données).
	Elles peuvent être : statistique, par graphes, linguistique ou par apprentissage automatique.
\end{itemize}

\subsubsection{Approche statistique}

Dans cette approche, les unités du texte (généralement, des phrases) sont attribuées un score de pertinence selon des critères statistiques.
Parmi les critères, on peut utiliser la fréquence des mots. 
L'équation \ref{eq:ats-tfidf} représente un exemple d'un score d'une phrase $s_i$ basé sur TF-IDF. 
\begin{equation}\label{eq:ats-tfidf}
Score_\text{TF-IDF}(s_i) = \sqrt{\sum\limits_{w_{ik} \in s_i} (\text{TF-IDF}(w_{ik}))^2}
\end{equation}
La position de la phrase est un bon critère de sa pertinence ; les phrases au début et à la fin parient plus importantes.
L'équation \ref{eq:ats-pos} représente un score attribué à la phrase $s_i$ d'un document $D$ en utilisant sa position.
\begin{equation}\label{eq:ats-pos}
Score_\text{pos}(s_i) = \max (\frac{1}{i}, \frac{1}{|D| - i + 1})
\end{equation}
La taille de la phrase peut être utilisée comme critère ; on peut fixer une taille maximale ou minimale pour les phrases acceptées dans le résumé.
L'équation \ref{eq:ats-taille} représente un score attribué à la phrase $s_i$ en utilisant une taille minimale $L_{min}$.
\begin{equation}\label{eq:ats-taille}
Score_\text{taille}(s_i) = \left\lbrace 
\begin{array}{lll}
0 & si & (L_i \geq L_{min}) \\
\frac{L_i - L_{min}}{L_{min}} & sinon & \\
\end{array}
\right.
\end{equation}
Les mots des titres et des sous-titres sont des indicateurs de la pertinence d'une phrase. 
Soit $T$ un titre, le score d'une phrase $s_i$ peut être calculé en utilisant la fréquence des mots dans le document $tf$ selon l'équation \ref{eq:ats-titre}.
\begin{equation}\label{eq:ats-titre}
Score_{titre}(s_i) = \frac{\sum_{e \in T \bigcap s_i}{\frac{tf(e)}{tf(e)+1}}}
{\sum_{e \in T}{\frac{tf(e)}{tf(e)+1}}}
\end{equation}
Il y a plusieurs formulations de ces critères ; pas seulement celles mentionnées ici. 
Aussi, il existe d'autres critères comme : Centroid, Frequent itemsets, Analyse sémantique latente, etc.

Parmi les méthodes statistiques, on peut mentionner la méthode TCC (\textit{Topic Clustering and Classification}) proposée dans \cite{13-aries-al} implémentée dans un système appelé AllSummarizer\footnote{AllSummarizer : \url{https://github.com/kariminf/allsummarizer}}.
L'architecture de cette méthode est illustrée dans la figure \ref{fig:ats-tcc}.
L'idée est de regrouper les phrases comme sujets en utilisant une méthode de regroupement ; une phrase peut appartenir à plusieurs sujets. 
Pour ce faire, la similarité cosinus et un seuil de regroupement ($Th$) sont utilisés.
Chaque phrase est attribuée un score qui indique combien elle peut représenter tous les sujets du document. 
Naive Bayes est utilisé pour apprendre les propriétés de chaque sujet et noter les phrases.
%
On utilise un ensemble des caractéristiques $f$ sur la phrase : TF (Uni-gramme, Bi-gramme), la position, la taille avent et après le pré-traitement.
Une phrase $s_i$ est jugée comme représentative du cluster $c_j$ en utilisant une caractéristique $f_k$ selon le score indiqué dans ll'équation \ref{eq:ats-tcc-score-si}.
\begin{equation}\label{eq:ats-tcc-score-si}
Score(s_i , c_j , f_k ) = 1 + \sum_{\phi \in s_i} {P(f_k=\phi | s_i \in c_j)}
\end{equation}
Une phrase $s_i$ est jugée pertinente si elle peut représentée tous les sujets (clusters) $c$ selon toutes les caractéristiques $f$.
Le score d'une phrase en se basant sur cette intuition peut être formulé comme l'équation \ref{eq:ats-tcc-score-all}.
\begin{equation}\label{eq:ats-tcc-score-all}
Score(s_i , \bigcap_{j} c_j , F) = \prod_{j} \prod_{k} Score(s_i , c_j , f_k )
\end{equation}

\begin{figure}[!ht]
	\centering
	\hgraphpage[.8\textwidth]{tcc-arch_noir.pdf}
	\caption{Architecture de la méthode TCC pour le résumé automatique statistique \cite{13-aries-al}}
	\label{fig:ats-tcc}
\end{figure}

Une autre méthode statistique est celle proposée dans \cite{15-oufaida-al}.
Les mots sont représentés en vecteurs en utilisant un \keyword[E]{embedding} pré-entraîné (Polyglot).
Premièrement, on commence par clustering des phrases pour extraire les sous-sujets du texte. 
Pour chercher le plus similaire à un mot $w_i$ dans une phrase $S_2$, on utilise une similarité entre les vecteurs (cosinus par exemple) comme indiqué dans l'équation \ref{eq:ats-oufaida-match}.
\begin{equation}\label{eq:ats-oufaida-match}
Match(w_i | S_2) = \arg\max_{w_j \in S_2} sim(Rep(w_i), Rep(w_j))
\end{equation}
Pour calculer la similarité entre deux phrases $S_1$ et $S_2$, on peut utiliser la fonction de matching comme précisé dans l'équation \ref{eq:ats-oufaida-sim}.
\begin{equation}\label{eq:ats-oufaida-sim}
Sim(S_1, S_2) = \frac{\sum_{w_i \in S_1} Match(w_i | S_2) + \sum_{w_j \in S_2} Match(w_j | S_1)}{|S_1| + |S_2|}
\end{equation}
Cette similarité est utilisée pour regrouper les phrases en sujets. 
Afin d'attribuer un score à une phrase, on utilise les scores des termes qui lui composent. 
Le score des termes est calculé selon une méthode appelée mRMR (Minimum Redundancy Maximum Relevance). 
On commence par la création d'une représentation termes/phrases indiquant la fréquence des termes dans chaque phrase (voir chapitre 6). 
Dans ce cas, un terme est représenté par un vecteur des phrases. 
Étant donné deux termes $X$ et $Y$, on calcule l'information mutuelle comme indiqué dans l'équation \ref{eq:ats-oufaida-mutinf}.
\begin{equation}\label{eq:ats-oufaida-mutinf}
I(X, Y) = \sum\limits_{x \in X} \sum\limits_{y \in Y} p(x, y) \log \frac{p(x, y)}{p(x) p(y)}
\end{equation}
Les probabilités sont calculées par rapport aux clusters des phrases.
La pertinence d'un terme $T_i$ peut être calculée en se basant sur son information mutuelle avec la représentation cluster/phrases $H$ par l'équation \ref{eq:ats-oufaida-rel}.
\begin{equation}\label{eq:ats-oufaida-rel}
Pertinence(T_i) = I(T_i, H)
\end{equation}
Sa redondance est calculée par rapport à au résumé $R$ par l'équation \ref{eq:ats-oufaida-red}.
\begin{equation}\label{eq:ats-oufaida-red}
Redondance(T_i) = \frac{1}{|R|} \sum\limits_{T_j \in R} I(T_i, T_j)
\end{equation}
Le score final d'un terme peut être calculé en utilisant deux méthodes : MID ou MIQ (voir l'équation \ref{eq:ats-oufaida-termscore}).
\begin{equation}\label{eq:ats-oufaida-termscore}
MID \equiv \max_{t \in T} Pertinence(t) - Redondance(t), \quad
MIQ \equiv \max_{t \in T} Pertinence(t) / Redondance(t)
\end{equation}
Le vecteur contenant tous les scores des termes est référencé par $V_{mRMR}$. 
Le score d'une phrase est sa similarité avec ce vecteur ; la plus similaire est ajoutée au résumé. 
Lors de l'ajout, on décrémente les scores des termes qui composent cette phrase dans le vecteur $V_{mRMR}$. 

\subsubsection{Approche par graphes}

Dans cette approche, les phrases sont représentés comme nœuds d'un graphe $G(V, A)$ ; les arcs représentent les similarités entre ces phrases. 
Une méthode pour sélectionner les phrases pertinentes est d'utiliser les \optword{propriétés du graphe}. 
Parmi les propriétés, on peut citer ``Bushy paths" qui note une phrase $s_i$ en se basant sur le nombre des arcs qui la relient avec les autres phrases.
Ce score est exprimé dans l'équation \ref{eq:ats-graph-bushy}.
\begin{equation}\label{eq:ats-graph-bushy}
Score_{\#arcs}(s_i) = |\{ s_j : a(s_i, s_j) \in A / s_j \in S, s_i \neq s_j \}|
\end{equation}
Une autre propriété s'appelle ``Aggregate Similarity" qui note une phrase $s_i$ en se basant sur la somme des poids de ses arcs comme indiqué dans l'équation \ref{eq:ats-graph-aggregate}.
\begin{equation}\label{eq:ats-graph-aggregate}
Score_{aggregate}(s_i) = \sum\limits_{(s_i, s_j) \in E} sim(s_i, s_j)
\end{equation}
Une autre approche est d'utiliser des \optword{méthodes itératives} en mettant à jours les scores des nœuds par rapport aux voisins jusqu'à arriver à un état d'équilibre. 
Parmi les méthodes qui se basent sur cette technique, on peut citer TextRank \cite{04-mihalcea-tarau}. 
Le poids $WS$ d'un nœud $V_i$ est calculé selon l'équation \ref{eq:ats-graph-textrank1}.
\begin{equation}\label{eq:ats-graph-textrank1}
WS(V_i) = ( 1 - d) + d * \sum\limits_{V_j \in In(V_i)} \frac{w_{ji}}{\sum\limits_{V_k \in Out(V_j)} w_{jk}} WS(V_j)
\end{equation}
le facteur $ d $ est fixé en général à $ 0.85 $.
Le poids $w_{ij}$ d'un arc entre les phrases $S_i$ et $S_j$ est calculé selon l'équation \ref{eq:ats-graph-textrank2}.
\begin{equation}\label{eq:ats-graph-textrank2}
w_{ij} = \frac{|\{w_k \text{ / } w_k \in S_i \text{ and } w_k \in S_j\}|}{\log(|S_i|) + \log(|S_j|)}
\end{equation}

Les graphes ne sont pas utilisés seulement pour attribuer des scores aux phrases, mais aussi on peut les utiliser pour faire une pré-sélection.
Dans la méthode SSF-GC (\textit{Sentence statistical features - graph cumulative score})\cite{21-aries-al}, on commence par noter les phrases selon des caractéristiques statistiques (voir la figure \ref{fig:ats-gc}). 
Ensuite, on crée un graphe en utilisant les similarités entre ces phrases. 
Le graphe est simplifié afin de garder les phrases les plus probables à être sélectionnées dans le résumé. 
Ensuite, on améliore le score des phrases en utilisant les propriétés du graphe.

\begin{figure}[!ht]
	\centering
	\hgraphpage[.5\textwidth]{gc-archi_noir.pdf}
	\caption{Architecture de la méthode SSF-GC pour le résumé automatique statistique \cite{21-aries-al}}
	\label{fig:ats-gc}
\end{figure}

Un graphe $G(V, E)$ est construit en utilisant les phrases et leurs similarités cosinus sur les fréquences des mots.
Le graphe est simplifié par l'élimination des nœuds faibles. 
Un nœuds est jugé faible s'il ... la propriété exprimée par l'équation \ref{eq:ats-gc-nfaible}.
$MImpN(v_i)$ est le nombre des voisins les plus importants ; ceux qui ont une similarité avec $v_i$ supérieure à un seuil donné $Threshold$.
\begin{equation}\label{eq:ats-gc-nfaible}
noeud\_faible(v_i) = ( \sum_{(v_i, v_j) \in E} w_{ij} < \frac{1}{MImpN(v_i)} )
\end{equation}
Les arcs faibles sont, aussi, éliminés selon l'équation \ref{eq:ats-gc-afaible}
\begin{equation}\label{eq:ats-gc-afaible}
arc\_faible(v_i, v_j) = ( w_{ij} < \frac{Threshold}{MImpN(v_i)})
\end{equation}
Une fois le graphe est simplifié, on score chaque phrase selon des caractéristiques statistiques $f_i \in F$ (vues précédemment). 
Ces scores sont considérés comme des probabilités, d'où la probabilité totale est la multiplication entre leurs probabilités comme indiqué dans l'équation \ref{eq:ats-gc-ssf}.
\begin{equation}\label{eq:ats-gc-ssf}
SSF(s_i/ F) = \prod_{f_i \in F} score(s_i/f_i)
\end{equation}
Une des méthodes pour améliorer le score $SSF$ d'une phrase $s_i$ est d'utiliser la somme des scores $SSF$ de ses voisins pondérés par les similarités, comme indiqué dans l'équation \ref{eq:ats-gc-gc1}.
\begin{equation}\label{eq:ats-gc-gc1}
GC1(s_i) = SSF(s_i) + \sum\limits_{(s_i, s_j) \in E} sim(s_i, s_j) * SSF(s_j)
\end{equation}
Le graphe peut être aussi utilisé lors de l'ajout d'une phrase au résumé. 
Une des méthodes d'extraction proposées, il y a une qui minimise l'ordre $ord$ de la similarité de la phrase $s_i$ à ajouter avec la dernière phrase ajouté $dernier_{e4}$ afin de réduire la redondance.
Elle vise à minimiser l'ordre inverse $iord$ du score $GC$ de la phrase, et donc maximiser son score de pertinence.
L'équation \ref{eq:ats-gc-ext4} représente la méthode du calcul de la phrase à ajouter au résumé.
\begin{equation}\label{eq:ats-gc-ext4}
suiv_{e4}  =  \arg\min\limits_i (iord\ gc(s_i) + ord\ sim(dernier_{e4}, s_i)) \text{ où } (dernier_{e4}, s_i) \in E
\end{equation}

\subsubsection{Approche linguistique}

On peut utiliser une liste des mots qui sont pertinents au sujet, comme ``significant", ``impossible", etc. 
Ces mots sont appelés \optword{Mots de sujet} et peuvent être utilisés pour noter les phrases. 
Dans \cite{69-edmundson}, on a défini deux listes : Bonus (mots positivement pertinents) et Stigma (mots négativement pertinents).
Le score d'une phrase $s_i$ en se basant sur ces deux listes est indiqué dans l'équation \ref{eq:ats-edmundson}.
\begin{equation}\label{eq:ats-edmundson}
Score_{cue}(s_i) = \sum_{w \in s_i}{cue(w)}
\text{ où }
cue(w) = \left\lbrace 
\begin{array}{ll}
b > 0 & \text{si } (w \in Bonus) \\
\delta < 0 & \text{si } (w \in Stigma) \\
0 & sinon 
\end{array} 
\right. 
\end{equation}

Une autre méthode est d'utiliser des \optword{Indicateurs} qui sont des structures qui impliquent que la phrase les contenant a une chose importante à propos du sujet.
Par exemple, \expword{the principal aim of this paper is to investigate ...}. 
La figure \ref{fig:paice-template} représente un patron défini par \cite{81-paice} afin de noter les phrases. 
\keyword{[x]} veut dire qu'il existe x mots entre ce mot et le mot précédent. 
Le score d'une phrase est augmenté avec la valeur \keyword{+y}.
Les mots optionnels sont annotés par \keyword{?}.

\begin{figure}[!ht]
	\centering
	\hgraphpage[.7\textwidth]{paice-template.pdf}
	\caption{Un exemple d'un patron simplifié \cite{81-paice}.}
	\label{fig:paice-template}
\end{figure}

D'autres méthodes utilisent des anaphores ou des représentations sémantiques (Ex. Wordnet) (\optword{Co-référence}) afin d'améliorer les scores des phrases. 
La \optword{structure rhétorique} peut être utilisée pour noter les phrases ou des syntagmes.


\subsubsection{Approche par apprentissage automatique (ML)}

En utilisant des \optword{caractéristiques} sur les phrases, le document, etc., on peut apprendre à résumer un document en utilisant un algorithme de \ac{ml}.
On peut soit régler des hyper-paramètres comme les poids des caractéristiques pour le score des phrases. 
Aussi, on utiliser l'apprentissage pour décider si une unité (phrase) appartient au résumé ou non (problème de classement). 
ML2ExtraSum\footnote{ML2ExtraSum \url{https://github.com/kariminf/ML2ExtraSum}} \cite{2020-aries} est un système qui se basent sur des caractéristiques définies manuellement afin d'estimer le score ROUGE1 (une métrique pour l'évaluation des résumés).
La figure \ref{fig:ats-ml2es} représente l'architecture de ce système.
En entrée, on a plusieurs caractéristiques comme une liste des TF des mots de la phrase ($sent\_tf\_seq$), une liste des TF des mots du document ($doc\_tf\_seq$), la taille de la phrase ($sent\_size$), etc.
Ces caractéristiques sont transformées en utilisant un module de transformation configurable. 
Par exemple, on peut transformer les listes en scalaires. 
Un bloc de réseau de neurones à propagation avant est utilisé pour détecter la langue (en réalité, il va attribuer à chaque document un vecteur : une forme de clustering).
Autres blocs sont utilisés pour calculer le score de la phrase en utilisant des critères : TF, Similarité, Taille et Position. 
Un bloc final est utilisé inférer le score ROUGE1 en utilisant les scores intermédiaires. 

\begin{figure}[!ht]
	\centering
	\hgraphpage[.7\textwidth]{ml2es-archi_noir.pdf}
	\caption{Architecture du système ML2ExtraSum qui utilisent des caractéristiques avec ML \cite{2020-aries}}
	\label{fig:ats-ml2es}
\end{figure}

Une autre approche est d'identifier les concepts principaux à partir des documents et les liens entre eux pour avoir une hiérarchie. 
On appelle ça \optword{Bayesian topic models}. 
Dans \cite{06-daumeiii-marcu}, on essaye de générer des résumés en utilisant des requêtes utilisateurs (voir la figure \ref{fig:ats-daumeii-marcu}). 
Pour ce faire, on essaye d'apprendre des modèles bayésiens à partir d'un ensemble $D$ de $K$ documents et un autre ensemble $Q$ de $J$ requêtes sur ces documents. 
Un modèle de langue général de l'anglais $P^G$ est entraîné sur des documents génériques.
Un autre modèle de langue des requêtes $P^Q$ est entraîné sur $Q$. 
Le troisième modèle de langue $P^D$ est entraîné sur les documents $D$ ; un modèle d'arrière-plan.
$r[K, J]$ est une matrice booléenne, qui a la valeur $1$ si le document $d \in D$ est pertinent à la requête $q \in Q$. 
Chaque mot $w_{dsn}$ d'un document $d$, d'une phrase $s$ et ayant la position $n$ a une variable cachée $z_{dsn}$ qui est un vecteur de taille $K+J+1$. 
Ce vecteur contient un seul $1$ et le reste de ces éléments sont des $0$.
Il indique de quel document ce mot a été généré : d'un des $K$ documents de $D$, d'un des $J$ requêtes de $Q$ ou d'un document générique.
Du même, pour chaque phrase $s$ d'un document $d$, oon attribue un vecteur $\pi_{ds}$ de taille $K+J+1$. 
Ce vecteur contient le degré de croyance que la phrase a été générée d'un des documents $D$, $Q$ ou l'anglais général.
Il peut être utilisé pour décider la phrase qui est générée par $D$ et $Q$ et pas par un document quelque ou par seulement l'un des deux.

\begin{figure}[!ht]
	\centering
	\hgraphpage[.4\textwidth]{btm-daumeiii_.pdf}
	\caption{Représentation du réseau bayésien utilisé dans le résumé automatique par \citet{06-daumeiii-marcu}}
	\label{fig:ats-daumeii-marcu}
\end{figure}

Une autre approche est d'utiliser des techniques de \optword{deep learning}. 
Le système NAMAS\footnote{NAMAS : \url{https://github.com/facebookarchive/NAMAS}} \cite{15-rush-al} utilise un réseau de neurone récurrent afin de générer un résumé à partir d'un petit texte. 
La figure \ref{fig:ats-namas} représente l'architecture de système qui sert à générer un résumé abstractif. 
Elle contient aussi un exemple d'un résumé généré à partir d'une phrase avec le degré d'attention pour chaque mot.
La partie (a) représente un décodeur qui cherche le mot prochain du résumé $y_{i+1}$ sachant la phrase en entrée $x$ et des mots déjà générés pour le résumé $y_c \equiv [y_{i-c+1},\ldots, y_i]$. 
La partie (b) représente un encodeur avec attention. 
Le mécanisme est similaire à celui présenté dans la traduction automatique.

\begin{figure}[!ht]
	\centering
	\hgraphpage[.35\textwidth]{2015-rush-al_.pdf}
	\hgraphpage[.35\textwidth]{2015-rush-al-exp_.pdf}
	\caption{Architecture du système NAMAS utilisant deep learning pour le résumé automatique abstractif \cite{15-rush-al}}
	\label{fig:ats-namas} 
\end{figure}

Une autre méthode pour entraîner un système de résumé automatique est d'utiliser \optword{reinforcement learning}. 
Il s'agit de l'utilisation des actions et des récompenses pour entrainer un système à générer des résumés.
La figure \ref{fig:ats-narayan} représente l'architecture du système de résumé proposé par \citet{18-narayan-al} basé sur l'apprentissage par renforcement.
Étant donné un document $D$ avec $n$ phrases, on veut extraire $m < n$ phrases pour construire le résumé. 
La sélection des phrases peut être formulée comme un problème de classement : classer une phrase comme appartenant au résumé $1$ ou non $0$.

\begin{figure}[!ht]
	\centering
	\hgraphpage[.85\textwidth]{narayan-al_.pdf}
	\caption{Architecture du système de \citet{18-narayan-al} utilisant l'apprentissage par renforcement}
	\label{fig:ats-narayan}
\end{figure}

On commence par décrire les deux encodeurs : phrases et document et l'extracteur (décodeur).
La phrase est encodée en utilisant des CNN sur la matrice contenant les mots (dans l'exemple, 7 mots encodés avec des vecteurs de 4 éléments). 
On utilise des filtres de déférentes tailles $h$ (ici, $4$ en bleu et $2$ en rouge) multiple fois (ici, on a utilisé $3$ différents filtres de même taille). 
Chaque filtre crée un vecteur $f \in \mathbb{R}^{k-h+1}$ où $k$ est le nombre des mots dans la phrases (ici, le premier vecteur est de taille $7-4+1 = 4$, le deuxième $7-2+1 = 6$). 
Le vecteurs résultats est passé par un Max-Pool pour avoir une seule valeur. 
Dans l'exemple, on a utilisé trois filtres de taille $4$ et trois de taille $2$, donc on aura deux vecteurs de  taille $3$ qui sont concaténés à un seul vecteur représentant la phrase.
Afin d'encoder le document, on passe les phrases par un réseaux récurrent LSTM en commençant par la représentation de la première phrase.
L'extracteur est un LSTM qui prend une phrase $s_i$ et estime une prédiction $y_i$ en prenant en compte la représentation du document : $p(y_i|x_i, D)$.

Lors de l'entrainement, on peut sélectionner les $m$ phrases d'un document $D$ en utilisant la métrique ROUGE-1 avec un résumé manuelle comme phrases de références.
On va annoter les phrases du document comme $y = y_1 \ldots y_n$ où $y_i \in \{0, 1\}$ ; les phrases de références ont un label de $1$. 
Les système peut être vu comme un agent qui utilise une politique $p(y_i|s_i, D, \theta)$ où $\theta$ sont les paramètres du modèle. 
Il génère des labels $\hat{y} = \hat{y}_1 \ldots \hat{y}_n$ où le résumé automatique est généré à partir des phrases les plus probables. 
L'agent est attribué une récompense (reward) $r$ indiquant comment le résumé automatique est similaire à celui manuelle. 
Cette récompense est formulée comme la moyenne des F1-scores des métriques ROUGE-1, ROUGE-2 et ROUGE-L. 
L'agent est mis à jours en essayant de minimiser l'espérance négative de la récompense comme indiqué dans l'équation \ref{eq:ats-narayan-error} où $p_\theta$ veut dire $p(y|D, \theta)$.
\begin{equation}\label{eq:ats-narayan-error}
L(\theta) = - \mathbb{E}_{\hat{y} \sim p_\theta} [r(\hat{y})]
\end{equation}
Le gradient peut être calculé comme dans l'équation \ref{eq:ats-narayan-grad}.
\begin{align}
\bigtriangledown L(\theta) & = - \mathbb{E}_{\hat{y} \sim p_\theta} [r(\hat{y}) \bigtriangledown \log p(\hat{y}|D, \theta)] \nonumber \\
& \approx - r(\hat{y}) \bigtriangledown \log p(\hat{y}|D, \theta) \nonumber \\
& \approx - r(\hat{y}) \sum_{i=1}^{n} \bigtriangledown \log p(\hat{y}_i|s_i, D, \theta) \label{eq:ats-narayan-grad}
\end{align}

%===================================================================================
\section{Questions-Réponses}
%===================================================================================

Un système de questions-réponses est un système interactif qui génère des réponses aux questions des utilisateurs.
La motivation derrière un tel système est claire : aider les utilisateurs à trouver des réponses.
La figure \ref{fig:qr-classif} représente une classification des systèmes de questions-réponses. 
Un tel système peut être générique (répondre à n'importe quelle question) ou spécifique à un domaine (la médecine est parmi les domaines les plus ciblés).
Il existe plusieurs approches pour implémenter un tel système : par \ac{ri}, à base de connaissance ou à base des modèles de langage.
Ces trois approches vont être présentées dans cette section avec quelques méthodes comme exemples.

\begin{figure}[!ht]
	\centering
	\hgraphpage[.8\textwidth]{qa-classif_noir.pdf}
	\caption{Classification des systèmes de Questions/Réponses}
	\label{fig:qr-classif}
\end{figure}

\subsubsection{Approche par RI}

Dans l'approche par \ac{ri}, on utilise un système de recherche d'information afin de récupérer les documents pertinents à la requête, ensuite on extrait le passage contenant la réponse.
La figure \ref{fig:qa-ri} représente une architecture un système de questions/réponses par \ac{ri}. 
Le système contient trois modules principaux : traitement de la requête, recherche des documents/passages et extraction de la réponse. 

\begin{figure}[!ht]
	\centering
	\hgraphpage[.8\textwidth]{qa-ri_.pdf}
	\caption{Architecture d'un système de questions/réponses par RI \cite{2019-jurafsky-martin}}
	\label{fig:qa-ri}
\end{figure}

Dans le traitement de la requête, on applique deux tâches : formulation de la requête et détection du type de réponse. 
La formulation de la requête concerne les tâches vues dans le chapitre 2 : Séparation des mots, Suppression des mots vides et Radicalisation. 
Le résultat est un ensemble des mots clés qui sont utilisés dans la recherche des documents. 
La réponse peut être : une personne, une place, une organisation, une cause, etc. 
Détecter le type de la réponse attendue va guider le système vers la bonne réponse (à extraire). 
Cela peut être accompli en utilisant une taxonomie comme Wordnet.

La recherche des documents se fait en utilisant les mots clés et l'index inversé des documents. 
Une fois les documents les plus pertinents (en se basant sur un score) sont retournés, on les divisent en passages (paragraphes ou phrases). 
Afin de rechercher les passages, on peut appliquer la détection des entités nommées et filtrer ceux qui ne contiennent pas le type de la réponse.
Dans ce cas, la recherche d'une cause plutôt du'une entité nommées sera plus difficile. 
Une autre solution est d'utiliser un algorithme d'apprentissage automatique afin de noter les passages. 
Parmi les caractéristiques qu'on puisse utilisées : nombre des entités nommées du type recherché, nombre des termes de la question, la séquence la plus longue similaire à la question, le rang du document, etc.

Une fois les passages pertinents sont identifiés, on passe à l'extraction de la réponse qui est une partie de ces passages. 
Une approche est d'utiliser un algorithme d'apprentissage automatique afin de détecter la réponse. 
Parmi les caractéristiques qui peuvent être utilisées : type de réponse et celui du syntagme, les mots clés de la question, la nouveauté (est ce qu'il y a un mot qui n'existe pas dans la question), la ponctuation (si la réponse est suivie par un virgule, point, etc.). 
Aussi, on peut utiliser des patrons comme \expword{\textless REP\textgreater comme \textless QES\textgreater; } qui permet de détecter des réponses comme ``expword{, des désordres développementaux comme l'autisme.}" à la question ``\expword{C'est quoi l'autisme ?}".


En utilisant les réseaux de neurones, on peut implémenter un système de questions-réponses dans le contexte de la tâche de lecture/compréhension. 
Pour chaque mot du passage, on calcule la probabilité d'être le début de la réponse et la fin de la réponse en se basant sur la question. 
C'est une tâche d'annotation des séquences qui peut être résolue en utilisant la technique \keyword[I]{IOB}.
Une méthode est celle de \citet{2017-chen-al}, illustrée dans la figure \ref{fig:qr-ri-chen}. 
On commence par encoder la question en passant les représentations \keyword[G]{GloVe} de ces mots par un réseaux récurrent Bi-LSTM, les vecteurs sont combinés en utilisant une somme pondérée. 
Afin d'encoder le passage, on utilise la représentation \keyword[G]{GloVe} des mots, un mécanisme d'attention basé sur les mots de la question, un indicateur si le mot a apparu dans la question ($0$ ou $1$) et des indicateurs de catégorie grammaticale et/ou le type de l'entité nommée.
Ces vecteurs sont passés par un réseaux Bi-LSTM afin d'extraire les embeddings des mots par rapport au passage. 
Chaque vecteur d'un mot du passage est comparé avec le vecteur de la requête en utilisant une similarité afin de décider si le mot appartient à la réponse.

\begin{figure}[!ht]
	\centering
	\hgraphpage{qa-bilstm-exp_.pdf}
	\caption{Extraction de la réponse par Bi-LSTM \cite{2019-jurafsky-martin}}
	\label{fig:qr-ri-chen}
\end{figure}

Nous avons vu dans le chapitre 6 que BERT peut être utilisé dans des tâche d'annotation des séquences.
Dans \cite{2018-devlin-al}, les auteurs passent la question et le passage au modèle pré-entraîné \keyword[B]{BERT} (voir la figure \ref{fig:qr-ri-devlin}).
On utilise deux représentations, de début (S) et de fin (E), qui sont multipliés par la représentation de chaque mot et passé par une fonction softmax sur l'ensemble de tous les mots afin d'avoir la probabilité de début et celle de fin.

\begin{figure}[!ht]
	\centering
	\hgraphpage[.5\textwidth]{qa-bert-exp_.pdf}
	\caption{Extraction de la réponse par BERT \cite{2019-jurafsky-martin}}
	\label{fig:qr-ri-devlin}
\end{figure}

\subsubsection{Approche par connaissance}

Lorsqu'on possède une base de données structurée, cette approche est la plus adéquate. 
Dans le cas d'une base de connaissance sous forme de graphe (BabelNet), on peut utiliser l'annotation sémantique (Entity linking). 
Ensuite, on doit déterminer le type de la relation recherchée (Ex. \expword{Place\_naissance}) afin de filtrer les entités.
S'il y a plusieurs relations comme réponse, on peut calculer la similarité entre la question et la réponse (par exemple, en utilisant les embeddings). 
Une autre approche est en appliquant l'analyse sémantique sur la question afin de générer une forme structurée comme : lambda calcul, SQL, SPARQL, etc.
Cette forme est utilisée afin d'interroger la base de données et récupérer la réponse.
Un exemple des requêtes et leurs formes logiques est donnée dans la figure \ref{fig:qr-conn}.

\begin{figure}[!ht]
	\centering
	\hgraphpage[.8\textwidth]{qa-logfrm-exp_.pdf}
	\caption{An exemple des formes logiques des questions \cite{2020-jurafsky-martin}}
	\label{fig:qr-conn}
\end{figure}

\subsubsection{Approche par modèles de langues}

Dans cette approche, on utilise un modèle de langage pré-entrainé. 
Ensuite, on le règle pour répondre les questions à partir du modèle (pas besoins de passages). 
Dans \cite{2020-roberts-al}, on a entrainé un modèle T5 sur la tâche de remplir un texte absent (voir la figure \ref{fig:qr-modele-t5}).
Ensuite, le modèle est réglé pour répondre aux questions sans saisir d'informations ou de contexte supplémentaires.

\begin{figure}[!ht]
	\centering
	\hgraphpage[.7\textwidth]{qa-t5_.pdf}
	\caption{T5 comme modèle de réponse aux questions \cite{2020-roberts-al}}
	\label{fig:qr-modele-t5}
\end{figure}

%===================================================================================
\section{Systèmes de dialogue}
%===================================================================================

\begin{figure}[!ht]
	\hgraphpage[.8\textwidth]{sd-classif_noir.pdf}
	\caption{Classification des systèmes de dialogue}
	\label{fig:sd-classif}
\end{figure}


\subsubsection{Orienté tâche : Frame-based}

	\begin{itemize}
	\item Frame (Cadre) : une structure contenant des slots à remplir et des questions prédéfinies pour chaque slot
	\item On ne pose que les questions dont le slot est vide
	\item Il y a la possibilité de remplir d'autres slots dans d'autre cadres. 
	Ex. \expword{Le slot RESERVATION\_DATE dans le cadre HOTEL\_RESERVATION à partir du slot ARRIVAL\_DATE du cadre FLIGHT\_RESERVATION}
	\item Appliquer \keyword{Détection d'intention} pour savoir quel cadre à utiliser
\end{itemize}

\begin{figure}
	\centering
	\hgraphpage[.6\textwidth]{sd-frame-exp_.pdf}
	\caption{Exemple d'un cadre pour programmer un vol. \cite{2020-jurafsky-martin}}
\end{figure}

orienté tâche - Frame-based (Réponses multiples)
\begin{figure}
	\centering
	\hgraphpage[.6\textwidth]{sd-frame-semgram-exp_.pdf}
	
	\hgraphpage[.8\textwidth]{sd-frame-parse-exp_.pdf}
	\caption{Exemple d'une grammaire sémantique et un arbre sémantique d'une phrase. \cite{2020-jurafsky-martin}}
\end{figure}

\subsubsection{Orienté tâche : Dialogue-State}

\begin{figure}
	\centering
	\hgraphpage[.7\textwidth]{sd-dialog-arch_.pdf}
	\caption{Architecture d'un système utilisant dialogue-state \cite{2016-williams-al}}
\end{figure}

orienté tâche - Dialogue-State (Actes du dialogue)
\begin{figure}
	\centering
	\hgraphpage[.5\textwidth]{sd-dialog-act-exp1_.pdf}
	
	\hgraphpage[.5\textwidth]{sd-dialog-act-exp2_.pdf}\vspace{-6pt}
	\caption{Des actes de dialogue ainsi qu'un exemple d'un dialogue d'un système de recommandation des restaurants appelé HIS \cite{2010-young-al}. Il est indiqué ce qui est permété d'être une entrée du système et une sortie de l'utilisateur \cite{2020-jurafsky-martin}}
\end{figure}

orienté tâche - Dialogue-State (Remplissage des slots)
\begin{itemize}
	\item Classifier la phrase par intention, domaine et slot
	\item Extraire les informations pour remplir les slots. Ex. \expword{En utilisant l'apprentissage automatique pour l'étiquetage des séquences}
\end{itemize}

\begin{figure}
	\centering
	\hgraphpage[.5\textwidth]{sd-dialog-remp-exp_.pdf}
	\caption{Exemple d'une architecture pour le remplissage des slots en utilisant BERT \cite{2020-jurafsky-martin}}
\end{figure}

orienté tâche - Dialogue-State (Autres composants)
\begin{itemize}
	\item Traqueur de l'état du dialogue
	\begin{itemize}
		\item Sauvegarder l'état des cadres (slots) et le dernier acte du dialogue
	\end{itemize}
	\item Politique du dialogue
	\begin{itemize}
		\item Déterminer l'action $A_i$ à prendre 
		
		$\hat{A}_i = \arg\max_{A_i \in A} P(A_i | Frame_{i-1}, A_{i-1}, U_{i-1})$
	\end{itemize}
	
	\item Générateur du texte 
	\begin{itemize}
		\item Générer du texte à partir d'un acte de dialogue
		\item On peut entraîner un encodeur/décodeur pour le faire
	\end{itemize}
\end{itemize}

\subsubsection{Chatbot à base des règles }

(Exemple : ELIZA \cite{1966-Weizenbaum})
\begin{itemize}
	\item Le plus célèbre chatbot est \keyword{ELIZA} : psychologue
	\item Une liste des patrons/transformations 
	
	\expword{\small [(.*) YOU (.*) ME]\textsubscript{[Patron]} \textrightarrow\ [WHAT MAKES YOU THINK I \$2 YOU?]\textsubscript{[Transformation]}}
	
	\expword{You hate me \textrightarrow\ WHAT MAKES YOU THINK I HATE YOU?}
	
	\item Les patrons sont liés à une liste des mots. Le mot qui score le plus dans la phrase va déclencher plusieurs patrons
	
	\item Parmi les, le patron le plus similaire à la phrase de l'utilisateur est utilisé
\end{itemize}

\subsubsection{Chatbot à base de la RI}

\begin{itemize}
	\item Un corpus des conversations $C$
	\item On prend le texte de l'utilisation comme une requête (question) $q$
	\item On cherche la réponse $r \in C$ qui est plus similaire à la requête $q$
	\[\text{Réponse}(q, C) = \arg\max_{r \in C} \frac{q . r}{|q| |r|}\]
	\item Pour calculer la similarité, on peut utiliser TF-IDF
	\item On peut aussi encoder la requête $q$ et la réponse $r$ en utilisant les embeddings
	\[h_q = BERT_Q(q)[CLS],\; h_r = BERT_R(r)[CLS]\]
	\[\text{Réponse}(q, C) = \arg\max_{r \in C} q . r\]
\end{itemize}

\subsubsection{Chatbot par génération du texte}

\begin{itemize}
	\item Utiliser un encodeur/décodeur comme dans la traduction automatique
	\[ \hat{r}_i = \arg\max_{w \in V} p(w| q, r_1, \ldots, r_{t-1}) \]
	
	\item Dans l'encodeur, introduire un contexte plus long
	
	\item On peut utiliser un modèle de langue, comme GPT, afin de l'entraîner sur des conversations
\end{itemize}

\begin{figure}
	\centering
	\hgraphpage[.7\textwidth]{sd-chatbot-encdec-exp_.pdf}
	\caption{Exemple d'un chatbot à base d'un encodeur/décodeur \cite{2020-jurafsky-martin}}
\end{figure}


%===================================================================================
\section{Analyse des sentiments}
%===================================================================================


\hgraphpage{sentiment-classif.pdf}

Analyse des sentiments : à base de connaissance
\begin{itemize}
	\item Identifier les mots indicateurs des sentiments
	\begin{itemize}
		\item En général, les adjectifs et les adverbes sont des bons indicateurs
		\item \optword{à base de dictionnaire} : en utilisant un dictionnaire comme Wordnet pour enrichir la liste des indicateurs (synonymes, antonymes, etc.) 
		\item \optword{à base d'un corpus} : en utilisant la co-occurrence avec les mots  de la liste originale pour l'enrichir 
	\end{itemize}
	\item Assigner des scores à ces mots (ou des étiquettes)
	\item Calculer le score total à  base des mots
	\begin{itemize}
		\item \textbf{granularité} : par phrases ou par documents 
		\item il faut faire attention à  la négation
		\item on peut utiliser la structure; par exemple \keyword{RST}
		\item Ex. \expword{Donner un score de -1 aux mots de polarité négative et +1 aux mots de polarité positive et faire la somme des polarités des mots d'une phrase}
	\end{itemize}
\end{itemize}

Analyse des sentiments : par apprentissage automatique
\begin{itemize}
	\item \optword{En utilisant des caractéristiques }
	\begin{itemize}
		\item Présence d'un mot et sa fréquence 
		\item Catégorie grammaticale
		\item Les mots et les syntagmes d'opinion
		\item Négation
	\end{itemize}
	\item \optword{En utilisant des embeddings}
	\begin{itemize}
		\item Utiliser les embeddings des mots pour apprendre la classe de sortie
		\item En utilisant BERT (l'apprentissage par transfert)
		\begin{itemize}
			\item Utiliser la sortie \keyword{[CLS]} pour estimer les classes
			\item Utiliser \keyword{[CLS]} avec un réseaux feedForward pour estimer les classes
			\item Utiliser les derniers états cachés des mots avec des configurations comme CNN, RNN, etc.
		\end{itemize}
	\end{itemize}
\end{itemize}

Analyse des sentiments : hybride (exemple \cite{18-bettiche-al}) 
\begin{figure}
	\centering
	\hgraphpage[.7\textwidth]{sent-bettiche-al_.pdf}
	\caption{Architecture hybride proposée par \cite{18-bettiche-al} pour détecter la polarité des messages en dialecte algérienne sur les réseaux sociaux}
\end{figure}

	\begin{itemize}
	\item Enrichissement du vocabulaire
	\begin{itemize}
		\item Détection des mots similaires (désignation d'un représentant)
		
		$Ratio = 1 - Levenstein(w_1, w_2)/(|w_1|+|w_2|)$
		
		Ex. \expword{Ratio(kolach, kollach) = 92\%, Ratio(kolach, khlasse) = 38\%.}
		
		\item Calculer l'orientation sémantique d'un mot $w$
		
		
		$SO(w) = \sum_{w_p \in V_p} PMI(w, w_p) - \sum_{w_n \in V_n} PMI(w, w_n)$
		
		$PMI (w_1, w_2) = \log \frac{p(w_1, w_2)}{p(w_1)*p(w_2)}$
	\end{itemize}
	\item Représentation de la polarité
	\begin{itemize}
		\item $polarite(w) = +1 \text{ SI } SO(w) > 0; -1 \text{ SI } SO(w) < 0$
		\item $polarite(w) = SO(w)$
	\end{itemize}
	\item Version avec règles : $polarite(msg) = Signe \sum_{w \in msg} polarite(w)$
	\item Version avec apprentissage automatique
	\begin{itemize}
		\item les polarités des mots comme caractéristiques
		\item ou la corrélation $PMI$ pour sélection des mots d'entrée
		\item plusieurs algorithmes sont utilisés : NB, arbres de décision, SVM, etc.
	\end{itemize}
\end{itemize}

Analyse des sentiments : hybride (exemple \cite{18-guellil-al}) 
\begin{figure}
	\centering
	\hgraphpage[.45\textwidth]{sent-guellil-al_.pdf}
	\caption{Architecture hybride proposée par \cite{18-guellil-al} pour l'analyse des sentiments des messages en arabizi sur les réseaux sociaux}
\end{figure}

%===================================================================================
\section{Lisibilité}
%===================================================================================

\hgraphpage{lisibilite-classif.pdf}

Lisibilité : Formule (Flesch-Kincaid Grade Level)
\[
206.835 - 1.015 (\frac{\text{\slshape mots totaux}}{\text{\slshape phrases totales}})
- 84.6 (\frac{\text{\slshape syllabes totales}}{\text{\slshape mots totaux}})
\]

\begin{center}
	\rowcolors{2}{lightblue}{lightyellow}\footnotesize
	\begin{tabular}{p{.15\textwidth}lp{.5\textwidth}}
		\rowcolor{darkblue}
		\bfseries\textcolor{white}{Score} && \bfseries\textcolor{white}{Difficulté}\\
		90-100 && Très facile à lire (Élève de 11 ans). \\
		80-90 && Facile à lire. \\
		70-80 && Plutôt facile à lire.\\
		60-70 && En clair (Élève de 13 ou 15 ans). \\
		50-60 && Plutôt difficile à lire. \\
		30-50 && Difficile à lire (Université). \\
		0-30 && Très difficile à lire (Diplôme universitaire). \\
	\end{tabular}
\end{center}

Lisibilité : Formule (Dale–Chall readability formula)
\[
0.1579 (\frac{\text{\slshape mots difficiles}}{\text{\slshape mots totaux}})
+ 0.0496 (\frac{\text{\slshape mots totaux}}{\text{\slshape phrases totales}})
\]

\begin{center}
	\rowcolors{2}{lightblue}{lightyellow}\footnotesize
	\begin{tabular}{p{.15\textwidth}lp{.5\textwidth}}
		\rowcolor{darkblue}
		\bfseries\textcolor{white}{Score} && \bfseries\textcolor{white}{Difficulté}\\
		\textless= 4.9 && Étudiant du 4ième. \\
		5-5.9 && Étudiant du 5ième et 6ième. \\
		6-6.9 && Étudiant du 7ième et 8ième.\\
		7-7.9 && Étudiant du 9ième et 10ième. \\
		8-8.9 && Étudiant du 11ième et 12ième. \\
		9-9.9 && Étudiant du 13ième et 15ième (collège). \\
	\end{tabular}
\end{center}

Lisibilité : Apprentissage
\begin{figure}
	\centering
	\hgraphpage[0.8\textwidth]{lisibilite-ML_.pdf}
	\caption{Pipeline d'estimation de difficulté de lecture avec apprentissage automatique \cite{2014-collins}}
\end{figure}


%===================================================================================
\section{Reconnaissance de la parole}
%===================================================================================


\hgraphpage{asr-classif.pdf}

\begin{figure}
	\centering
	\hgraphpage[.45\textwidth]{asr-arch_.pdf}
	\caption{Architecture d'un système de reconnaissance de paroles \cite{18-haridas}}
\end{figure}

\begin{itemize}
	\item \optword{Acoustique-Phonétique}
	\begin{itemize}
		\item Détecter les phonèmes et les transcrire
	\end{itemize}
	\item \optword{Reconnaissance des formes}
	\begin{itemize}
		\item Apprendre à détecter les différentes formes
		\item \optword{Par modèles} 
		\item \optword{Stochastique}
	\end{itemize}
	\item \optword{Intelligence artificielle}
	\begin{itemize}
		\item Fusion entre les deux approches précédentes
	\end{itemize}
\end{itemize}

Reconnaissance vocale : Extraction des caractéristiques (Échantillonnage)
\begin{itemize}
	\item Transformer les signaux sonores vers une séquence des vecteurs de caractéristiques acoustiques
	\item Chaque vecteur représente l'information du signal encodée dans une petite fenêtre de temps
	\item \optword{Échantillonnage}
	\begin{itemize}
		\item Il faut prendre 2 échantillons par cycle (pour capturer les parties positive et négative du signal)
		\item La fréquence des paroles sont inférieures à 10 KHz (sampling rate = 20 KHz)
		\item Dans la téléphonie, la fréquence est 4KHz (sampling rate = 8 KHz)
	\end{itemize}
\end{itemize}

Reconnaissance vocale : Extraction des caractéristiques (Quantification)
\begin{itemize}
	\item \optword{Quantification}
	\begin{itemize}
		\item Les amplitudes des échantillons sont stockées sous forme des entiers
		\item Soit 8 bits (-128 à 127) ou 16 bits (-32768 à 32767)
		\item La valeur d'un échantillon dans un temps $n$ est représentée comme $x[n]$
	\end{itemize}
\end{itemize}

Reconnaissance vocale : Extraction des caractéristiques (Fenêtrage)
	\begin{itemize}
	\item \optword{Fenêtrage}
	\begin{itemize}
		\item Utiliser une fenêtre pour capturer un ou une partie d'un phonème 
		\item Cette partie est appelée Cadre (frame)
		\item L'opération a deux paramètres : taille de la fenêtre (Window size) et décalage (Frame stride, shift, offset)
	\end{itemize}
\end{itemize}
\begin{figure}
	\centering
	\hgraphpage[.5\textwidth]{ASR-windowing-exp_.pdf}
	\caption{Exemple d'un fenêtrage \cite{2020-jurafsky-martin}}
\end{figure}

\begin{itemize}
	\item \optword{Fenêtrage} (suite)
	\begin{itemize}
		\item Pour extraire un cadre $y[n]$, on utilise un signal de la fenêtre $w[n]$ sur le signal initial $s[n]$
		\[y[n] = w[n] s[n]\]
		\item La plus simple est la fenêtre rectangulaire 
		
		$w[n] = \begin{cases}
		1 & \text{si } 0 \le n \le L-1 \\
		0 & \text{sinon }\\
		\end{cases}$
		\item Ceci va créer des problèmes avec l'analyse de Fourier 
		\item La solution est la fenêtre de Hamming
		
		$w[n] = \begin{cases}
		0.54 - 0.46 \cos (\frac{2\pi n}{L}) & \text{si } 0 \le n \le L-1 \\
		0 & \text{sinon }\\
		\end{cases}$
	\end{itemize}
\end{itemize}

\begin{figure}
	\centering
	\hgraphpage[.7\textwidth]{ASR-windowing2-exp_.pdf}
	\caption{Exemple d'un fenêtrage \cite{2020-jurafsky-martin}}
\end{figure}

Reconnaissance vocale : Extraction des caractéristiques (DFT)

\begin{itemize}
	\item \optword{Discrete Fourier Transform}
	\begin{itemize}
		\item Combien d'énergie un signal contient dans les différentes bandes de fréquence
		\[X[k] = \sum\limits_{n=0}^{N-1} x[n] e^{-j\frac{2\pi}{N} k n}\]
	\end{itemize}
\end{itemize}
\begin{figure}
	\centering
	\hgraphpage[.5\textwidth]{ASR-DFT-exp_.pdf}
	\caption{(a) une portion du signal du voyelle [iy] (b) son DFT \cite{2020-jurafsky-martin}}
\end{figure}

Reconnaissance vocale : Extraction des caractéristiques (Filtre de Mel)

\begin{itemize}
	\item \optword{Filtre de Mel}
	\begin{itemize}
		\item L'audition humaine n'est pas également sensible à toutes les bandes de fréquences
		\item Collecter les énergies à chaque fréquence en se basant sur l'échelle de Mel
		\[mel(f) = 1127 \ln (1 + \frac{f}{700})\]
	\end{itemize}
\end{itemize}
\begin{figure}
	\centering
	\hgraphpage[.5\textwidth]{ASR-mel-exp_.pdf}
	\caption{Un exemple du filtre de Mel \cite{2020-jurafsky-martin}}
\end{figure}

Reconnaissance vocale : Reconnaissance
\begin{figure}
	\centering
	\hgraphpage[.8\textwidth]{ASR-rec-exp_.pdf}
	\caption{Exemple de reconnaissance en utilisant un encodeur-décodeur \cite{2020-jurafsky-martin}}
\end{figure}

\begin{itemize}
	\item $p(y_1, \ldots, y_n) = \prod\limits_{i=1}^n p(y_i| y_1, \ldots, y_{i-1}, X)$
	
	\item $\hat{y}_i = \arg\max_{c \in V} p(c| y_1, \ldots, y_{i-1}, X)$
	
	\item clairement, ceci est un modèle de langue
	\item mais, il se peut que les données ne sont pas suffisants pour apprendre un bon modèle de langue
	\item Solution : intégrer un modèle de langue séparé
	
	\item $score(Y|X) = \frac{1}{|Y|_{car}} \log p(Y|X) + \lambda \log p_{LM}(Y)$
\end{itemize}


%===================================================================================
\section{Synthèse de la parole}
%===================================================================================

\hgraphpage{tts-classif.pdf}

\begin{figure}
	\centering
	\hgraphpage[.5\textwidth]{tts-arch_.pdf}
	\caption{Architecture d'un système de synthèse vocale \cite{2017-Hinterleitner}}
\end{figure}




\begin{discussion}



\end{discussion}

%=====================================================================
\ifx\wholebook\relax\else
% \cleardoublepage
% \bibliographystyle{../use/ESIbib}
% \bibliography{../bib/RATstat}
	\end{document}
\fi
%=====================================================================
